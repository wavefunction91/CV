%%%%%%%%%%%%%%%%%%%%%%%%%%%%%%%%%%%%%%%%%
% Long Professional Curriculum Vitae
% LaTeX Template
% Version 1.1 (9/12/12)
%
% This template has been downloaded from:
% http://www.latextemplates.com
%
% Original author:
% Rensselaer Polytechnic Institute (http://www.rpi.edu/dept/arc/training/latex/resumes/)
%
% Important note:
% This template requires the res.cls file to be in the same directory as the
% .tex file. The res.cls file provides the resume style used for structuring the
% document.
%
%%%%%%%%%%%%%%%%%%%%%%%%%%%%%%%%%%%%%%%%%

%----------------------------------------------------------------------------------------
%	PACKAGES AND OTHER DOCUMENT CONFIGURATIONS
%----------------------------------------------------------------------------------------

\let\latexnofiles\nofiles
\let\nofiles\relax
\documentclass[10pt]{res} % Use the res.cls style, the font size can be changed to 11pt or 12pt here

\usepackage{helvet} % Default font is the helvetica postscript font
%\usepackage{newcent} % To change the default font to the new century schoolbook postscript font uncomment this line and comment the one above
\usepackage{etaremune}
\newsectionwidth{0pt} % Stops section indenting

% Name shortcuts
\newcommand*\me[0]{{\bf Williams-Young, D. B.}}
\newcommand*\xsli[0]{Li, X.}
\newcommand*\ko[0]{Ko, J.}

\begin{document}

%----------------------------------------------------------------------------------------
%	YOUR NAME AND ADDRESS(ES) SECTION
%----------------------------------------------------------------------------------------

\name{David Williams-Young\\ \\} % Your name at the top

% If you don't want one of the addresses, simply remove all the text in the first or second \address{} bracket

\address{{\bf Academic Address} \\ 
Department of Chemistry \\ 
University of Washington\\ 
311A Bagley Hall (office)\\
Box 351700 (mail)\\
Seattle, WA, USA 98195-1700\\
dbwy@u.washington.edu} % Your address 1

\address{{\bf Permanent Address} \\ 
1737 NW 56th St \\ 
Apt 504 \\
Seattle, WA, USA 98107 \\ 
+1 (248) 245-1211} % Your address 2

%----------------------------------------------------------------------------------------

\begin{resume}

%----------------------------------------------------------------------------------------
%	OBJECTIVE SECTION
%----------------------------------------------------------------------------------------

\section{\centerline{OBJECTIVE}}

\vspace{8pt} % Gap between title and text

To obtain an academic position in chemical theory and computation.
%----------------------------------------------------------------------------------------
%	EDUCATION SECTION
%----------------------------------------------------------------------------------------

\section{\centerline{EDUCATION}} 

\vspace{8pt} % Gap between title and text

{\sl Doctor of Philosophy (Pursuant)}, Chemsitry \hfill Since September 2013 \\ 
University of Washington, Seattle, WA   \\ 
Advisor: Dr. Xiaosong Li
 
{\sl Bachelor of Science}, Chemistry \hfill May 2013  \\ 
Indiana University of Pennsylvania, Indiana, PA \\
Advisor: Dr. Jaeju Ko

%----------------------------------------------------------------------------------------
 
\vspace{0.2in} % Some whitespace between sections

%----------------------------------------------------------------------------------------
%	PROFESSIONAL EXPERIENCE SECTION
%----------------------------------------------------------------------------------------

\section{\centerline{PROFESSIONAL EXPERIENCE}} 

\vspace{8pt} % Gap between title and text

{\sl Graduate Research Assistant} \hfill July 2013 -- Present \\
University of Washington, Seattle, WA

{\sl Graduate Teaching Assistant} \hfill September 2013 -- Present \\
University of Washington, Seattle, WA 
\begin{itemize} \itemsep -2pt
%	\item Provided supplimentary lecture instruction (quiz section) for the General Chemistry course series
%    as well as full instruction for the associated lab coursework. I was responsible for
%    the evaluation of students' performance both in lecture course work (exams, homework assignments)
%    as well as the lab (lab reports).
  \item Provided supplimentary lecture instruction for the undergraduate General Chemistry course series
    as well as full instruction for the associated lab coursework. This job entailed full responsibility
    for two sections of thirty students during a supplimental lecture series referred to as a quiz section.
    I was responsible for the evaluatoin of students' performance both in lecture coursework (exams, homework assignments)
    as well as the lab (lab reports).
  \item Provided supplimetary lecture instruction for the final lecture course (in a series of three) of
    the undergraduate Organic Chemisty course series. Primarily, I was responsible for evaluating students' performance on
    Exams. In addition, I was responsible for making myself available for course related questions through
    structured office hours.
  \item Aided in the development of new course material for the second lecture course (in a series of two),
    of the graduate Quantum Chemistry course series. The developed course material focused on the practical
    development of basic quatnum chemical methods, such as Hartree-Fock and Density Functional Theory, using
    the MATLAB development environment. The students developed, from scratch, a working implementation
    of these methods with the aid of provided course material developed by my self and the primary course
    instructor. For this course, I also had the primary responsibility for evaluating students' progress
    through the grading of exams and assigned course work.
\end{itemize}

{\sl Undergraduate Research Assistant} \hfill September 2011 -- May 2013 \\
Indiana University of Pennsylvania, Indiana, PA

{\sl Chemistry Tutor} \hfill August 2010 -- May 2011 \\
Indiana University of Pennsylvania Disibility Services, Indiana, PA 
\begin{itemize}
  \item Provided supplimental course instruction for the General and Organic Chemistry
    series for students with disibilities. This primarily entailed meeting with students
    individually on a weekly basis to aid in their understanding of course material and 
    to ensure that they were not falling behind in coursework.
\end{itemize}

{\sl Information Technology Technition} \hfill Auguest 2009 -- January 2010 \\
Central Michigan University Information Technology, Mt. Pleasant, MI
\begin{itemize}
  \item Primarily responsible for solving network connectivity issues for incoming students
    at Central Michigan University.
\end{itemize}
%{\sl International Business Machines} \hfill January -- August 1990 \\
%General Products Division, Tucson, AZ \hfill (Co-op Assignment)
%\begin{itemize} \itemsep -2pt % Reduce space between items
%\item Developed new selection criteria for applicant screening and selection at GPD Tucson. Conducted job analysis, wrote criteria, identified skill codes for applicant tracking, established rater reliability. 
%\item Interviewed applicants for positions in Assembly, Warehousing, and Direct Customer Response. 
%\item Assistant Co-op Coordinator. Initiated and maintained computer tracking for Co-op program. Organized all co-op seminars and activities, co-op directory. 
%\item Representative on GPD Compensation Task Force. Prepared job descriptions, assigned corporate position code, and submitted for division approval. 
%\end{itemize}
%
%{\sl Rensselaer Polytechnic Institute, Troy, NY} \\[2pt]
%Professional Leadership Program, School of Management \hfill October 1988 -- March 1989 
%\begin{itemize} \itemsep -2pt % Reduce space between items
%\item Developed standardized interview to identify early management potential. Generated leadership dimensions and applicant assessment scale. 
%\item Completed 30 hours of interviewer training. Interviewed 75 candidates. 
%\end{itemize} 
%\vspace{-6pt} % Reduce space between positions at the same organization
%Department of Psychology \hfill May -- December 1989 
%\begin{itemize} 
%\item Trained engineering students in the development of interpersonal skills. Facilitated group interaction, provided individual feedback via application analysis papers. 
%\end{itemize}
% 
%{\sl New York State Department of Mental Health} \hfill September -- December 1989 \\
%Bureau of Management and Program Evaluation, Albany, NY 
%\begin{itemize}
%\item Conducted research to identify key evaluation factors in the statewide investigation of Intensive Care Facilities for disabled persons.
%\end{itemize}
%
%{\sl Colonie Center Shopping Mall} \hfill January -- June 1989 \\ Albany, NY 
%\begin{itemize} \itemsep -2pt % Reduce space between items
%\item Served as consultant for new management team in techniques for managing change. 
%\item Developed and administered organizational climate survey. 
%\item Facilitated management-employee feedback sessions. 
%\end{itemize}

%----------------------------------------------------------------------------------------

\vspace{0.2in} % Some whitespace between sections

%----------------------------------------------------------------------------------------
%	COMPUTER SKILLS SECTION
%----------------------------------------------------------------------------------------

\section{\centerline{COMPUTATIONAL PROFICIENCY}}

\vspace{8pt} % Gap between title and text

I am very proficient in the following computational areas:
\begin{itemize} \itemsep -2pt
  \item {\sl Programming Languages:} C/C++/C\#, FORTRAN 77/95/03, Java, Assembly Language
  \item {\sl Scripting Languages:} Python, Julia, R, MATLAB, Octave, C/Bash Shell
  \item {\sl Libraries / Paradigms:} OpenMP, TBB, Various MPI (OpenMPI, MPICH2, etc), OpenGL, CUDA,
  OpenACC, PThreads
  \item {\sl Software:} Gaussian, YASARA, PyMol, Mathematica, MATLAB
\end{itemize}

I have contributed to the development of the following software packages:
\begin{itemize} \itemsep -2pt
  \item Chronus Quantum Chemistry (ChronusQ) Software Pacakge ({\sl Principle Developer})
  \item Gaussian ({\sl Contributor})
  \vspace{-6pt}
  \begin{itemize}\itemsep -2pt
    \item Analytical hessians of time-dependent density functional theory
    \item Chebyshev expansion of the quantum propagator in real-time density functional theory
  \end{itemize}
\end{itemize}

%----------------------------------------------------------------------------------------

\vspace{0.2in} % Some whitespace between sections

%----------------------------------------------------------------------------------------
%	PUBLICATIONS SECTION
%----------------------------------------------------------------------------------------

\section{\centerline{PUBLICATIONS}} 

\vspace{15pt} % Gap between title and text
\begin{etaremune}
  \item Petrone, A.; Lingerfelt, D. B.; \me; \xsli;
        ``\emph{Ab Initio Transient Vibrational Spectral Analysis}"
	  \emph{J. Chem. Phys.}, \textbf{2016}, \emph{Submitted}.
  \item \me; Egidi, F.; \xsli;
	``\emph{Relativistic Two-Component Particle-Particle Tamm-Dancoff Approximation}"
	  \emph{J. Chem. Theor. Comp.}, \textbf{2016}, \emph{Submitted}.
  \item \me; Lingerfelt, D. B.; Petrone, A.; \xsli;
	``\emph{Ab Initio Surface Hoppings Dynamics Through Regions of Large Non-Adiabatic
	        Coupling within the Particle-Particle Tamm Dancoff Approximation (pp-TDA)}"
          \emph{J. Chem. Phys. Lett.}, \textbf{2016}, \emph{Submitted}.
  \item \me; Goings, J.; \xsli;
	``\emph{Accelerating Real-Time Time-Dependent Density Functional Theory 
	        with a Chebyshev Expansion of the Quantum Propagator}"
	  \emph{J. Chem. Theor. Comp.}, \textbf{2016}, \emph{Submitted}.
  \item Lingerfelt, D. B.; \me; Petrone, A; \xsli; 
        ``\emph{Direct ab Initio (Meta-)Surface-Hopping Dynamics}", 
        \emph{J. Chem. Theor. Comp.}, \textbf{2016}, 12(3), 935--945.
\end{etaremune}

\vspace{0.2in} % Some whitespace between sections

\section{\centerline{PRESENTATIONS}} 

\vspace{15pt} % Gap between title and text
\begin{etaremune}
  \item \me; Goings, J. J.; \xsli;
        ``\emph{Accelerating Real-Time Time-Dependent Density Functional Theory with a Chebyshev 
	  Expansion of the Quantum Propagator}";
	Theory and Applications of Computational Chemistry (TACC) 2016, Seattle, WA.
	\textbf{2016}. Poster.
  \item Egidi F.; \me; \xsli;
        ``\emph{Electronic Structure Methods for Relativistic Effects in Excited States}";
	Low Scaling and Unconventional Electronic Structure Theory (LUEST) 2016, Telluride, CO.
	\textbf{2016}. Poster.
  \item \me; Yang, W.; \xsli;
        ``\emph{Moving pase the particle-hole description of excited states: Affordable methodologies}";
	Chemical Congress of Pacific Basin Societies (PacifiChem) 2015, Honolulu, HI.
	\textbf{2015}. 
	PHYS: Recent Progress in Molecular Theory for Excited-state Electronic Structure and
	Dynamics. Oral Presentation.
%  \item Scalmani, G.; Frisch, M.; \me; \xsli;
%	``\emph{On the analytical second derivatives of the excited state energy in the 
%	        framework of Time-Dependant Density Functional Theory}";
%	248h American Chemical Society National Meeting \& Exposition, San Francisco, CA.
%	\textbf{2014}. COMP: Quantum Chemistry Symposium. Oral Presentation.
  \item \me; \ko; Ondrechen, M. J.;
	``\emph{Computational approach to the prediction of enzyme specificities}";
	245th American Chemical Society National Meeting \& Exposition, New Orleans, LA.
	\textbf{2013}. Sci-Mix Poster Session. Poster.
  \item \me; \ko; Ondrechen, M. J.;
	``\emph{Computational approach to the prediction of enzyme specificities}";
	245th American Chemical Society National Meeting \& Exposition, New Orleans, LA.
	\textbf{2013}. Division of Computers in Chemistry Poster Session. Poster.
  \item \me; \ko;
	``\emph{Prediction of Relitive Activities of Enzymes using Computed Chemical Properties}";
	American Chemical Society Student Memeber Symposium, Duquesne University. Duquesne, PA.
	\textbf{2012}. Poster.
  \item \me; \ko;
	``\emph{Prediction of Relitive Activities of Enzymes using Computed Chemical Properties}";
	Undergraduate Research Forum, Indiana University of Pennsylvania. Indiana, PA.
	\textbf{2012}. Poster.
  \item Ford, J.; \ko; Mintmier, B.; Machovia, T.; Kang, M.; Owens, A.; \me;
	``\emph{Undergraduate Research in Biomass Utilization Symposium}",
	Pennsylvania Association of the Council of Trustees Conference, Indiana University of Pennsylvania.
	Indiana, PA. \textbf{2011}. Oral Presentation.
\end{etaremune}
%----------------------------------------------------------------------------------------

\vspace{0.2in} % Some whitespace between sections

%----------------------------------------------------------------------------------------
%	MEMBERSHIPS SECTION
%----------------------------------------------------------------------------------------

\section{\centerline{MEMBERSHIPS}} 

\vspace{-5pt} % Reduce space between section title and contents

\begin{center}
Alpha Chi Sigma ($\mathrm{AX}\Sigma$), $\Gamma\mathrm{T}$ Chapter, Professional Chemistry Fraternity\\
Americal Chemical Society (ACS), Computers in Chemistry Division
\end{center}

%----------------------------------------------------------------------------------------

\vspace{0.2in} % Some whitespace between sections

%----------------------------------------------------------------------------------------
%	HONORS SECTION
%----------------------------------------------------------------------------------------

\section{\centerline{HONORS}} 

\vspace{-5pt} % Reduce space between section title and contents

\begin{center}
Early Bird Research Assistantship (EBRA), University of Washington (2013) \\
Excellence in Chemistry Graduate Fellowship Award (ECGFA), University of Washington (2013)\\
Provost Scholar, Indiana University of Pennsylvania (2013)
\end{center}

%----------------------------------------------------------------------------------------

\vspace{0.2in} % Some whitespace between sections

%----------------------------------------------------------------------------------------
%	INTERESTS SECTION
%----------------------------------------------------------------------------------------

\section{\centerline{INTERESTS}} 

\vspace{-5pt} % Reduce space between section title and contents

\begin{center}
World percussion, kayaking, hiking, climbing, Linux kernel development
\end{center} 

%----------------------------------------------------------------------------------------

\end{resume} 
\end{document}
