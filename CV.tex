%%%%%%%%%%%%%%%%%%%%%%%%%%%%%%%%%%%%%%%%%
% Long Professional Curriculum Vitae
% LaTeX Template
% Version 1.1 (9/12/12)
%
% This template has been downloaded from:
% http://www.latextemplates.com
% Original author:
% Rensselaer Polytechnic Institute (http://www.rpi.edu/dept/arc/training/latex/resumes/)
%
% Important note:
% This template requires the res.cls file to be in the same directory as the
% .tex file. The res.cls file provides the resume style used for structuring the
% document.
%
%%%%%%%%%%%%%%%%%%%%%%%%%%%%%%%%%%%%%%%%%

%----------------------------------------------------------------------------------------
%	PACKAGES AND OTHER DOCUMENT CONFIGURATIONS
%----------------------------------------------------------------------------------------

\let\latexnofiles\nofiles
\let\nofiles\relax
\documentclass[10pt]{res} % Use the res.cls style, the font size can be changed to 11pt or 12pt here

\usepackage{helvet} % Default font is the helvetica postscript font
%\usepackage{newcent} % To change the default font to the new century schoolbook postscript font uncomment this line and comment the one above
\usepackage{etaremune}
\usepackage{hyperref}

\newsectionwidth{0pt} % Stops section indenting

% Name shortcuts
\newcommand*\me[0]{{\bf Williams--Young,~D.~B.}}
\newcommand*\xsli[0]{Li,~X.}
\newcommand*\ko[0]{Ko,~J.}

\begin{document}

%----------------------------------------------------------------------------------------
%	YOUR NAME AND ADDRESS(ES) SECTION
%----------------------------------------------------------------------------------------

\name{David Williams--Young\\ \\} % Your name at the top

% If you don't want one of the addresses, simply remove all the text in the first or second \address{} bracket

\address{{\bf Academic Address} \\ 
Department of Chemistry \\ 
University of Washington\\ 
311A Bagley Hall (office)\\
Box 351700 (mail)\\
Seattle, WA, USA 98195-1700\\
dbwy@u.washington.edu} % Your address 1

\address{{\bf Permanent Address} \\ 
1737 NW 56th St \\ 
Apt 504 \\
Seattle, WA, USA 98107 \\ 
+1 (248) 245-1211} % Your address 2

%----------------------------------------------------------------------------------------

\begin{resume}

%----------------------------------------------------------------------------------------
%	OBJECTIVE SECTION
%----------------------------------------------------------------------------------------

\section{\centerline{OBJECTIVE}}

\vspace{8pt} % Gap between title and text

To obtain an academic position in chemical theory and computation.


%----------------------------------------------------------------------------------------
%       RESEARCH INTERESTS
%----------------------------------------------------------------------------------------

\section{\centerline{RESEARCH INTERESTS}}
\begin{itemize} \itemsep -2pt
  \item Development of low--scaling relativistic electronic structure methods to describe
  strongly correlated systems, such as transition metal complexes commonly found in
  the vicinity of the active sites of enzymes.
  \item Application of abstract mathematical paradigms, such as algebraic topology, to
  develop elegant and novel solutions to the problems that arise in ab initio electronic
  structure theory.
  \item Development of high--performance algorithms to treat the electronically non--adiabatic
  dynamics of quantum molecular systems.
\end{itemize}

%\section{\centerline{RESEARCH SUMMARY}}
%\begin{itemize} \itemsep -2pt
%  \item \textbf{Relativistic Excited State Electronic Structure Theory}, \emph{2014 -- Present}. 
%  A primary focus of my current and past research has been the development of efficient relativistic
%  excited state methods for electronic structure theory. Recently, I have extended the
%  particle--particle Tamm--Dancoff approximation, a method well known in \emph{ab initio}
%  nuclear structure theory, to two--component relativistic Hamiltonians. Using this method,
%  I was able to describe, with excellent accuracy, the fine--structure splitting of the excited
%  states of a set of atomic and molecular systems [\emph{J. Chem. Theor. Comp.}, \textbf{2016}, 12(11), 5379--5384].
%  I am also currently developing relativistic extensions for various other post-SCF methods such as
%  Coupled Cluster and Configuration Interaction.\\
%
%  \item \textbf{Molecular Dynamics}, \emph{2015 -- Present}. Recently, much of my research effort
%  has been directed to the development of novel and efficient electronic structure methods to describe
%  the electronic and nuclear dynamics of molecular systems in an \emph{ab initio} manner. I have developed
%  a revised algorithm for the evaluation of the quantum propagator of time--dependent density functional
%  theory (TDDFT) through the Chebyshev expansion of the matrix exponential [\emph{J. Chem. Theor. Comp.}, \textbf{2016}, 12(11) 5333--5338].
%  In principle, this development makes it possible to achieve linear--scaling electronic dynamics simulations
%  through the use of sparse matrix manipulation.\\
%  \\
%  Further, in collaboration with my colleagues in the Li research group, I have been involved with several 
%  exciting developments in the field of nonadiabatic excited state nuclear dynamics. The key to the usability
%  of any nonadiabatic molecular dynamics method is the underlying efficiency of the algorithms used to obtain
%  the required physical moieties, such as gradients and nonadiabatic couplings, throughout the simulation.
%  Much of my work in this field has been in the development of extremely efficient and scalable methods to
%  obtain these quantities. My work has facilitated the development of several methodological developments in the
%  field, including direct \emph{ab initio} rare event sampling through the meta--surface hopping method
%  [\emph{J. Chem. Theor. Comp.}, \textbf{2016}, 12(3), 935--945], and the analysis of transient vibrational signatures
%  to describe confomational changes through conformational changes 
%  [\emph{J. Chem. Phys. Comm.}, \textbf{2016}, 7, 4501--4508].
%\end{itemize}
%
%\section{\centerline{FUTURE RESEARCH STATEMENT}}
%  My plan for future research directions involves the combination of my previous research topics through
%  the application of relativistic hamiltonians to describe the nonadibatic behavior of molecular dynamics. To properly
%  treat nonadiabatic transitions between electronic states of differing spin multiplicities, such as intersystem crossings
%  and phosphorescence, a proper treatment of spin--orbit coupling is of utmost importance. While several approaches have
%  been to describe spin--orbit perturbatively in the non--relativistic regime, the only method to properly describe 
%  the effects of spin--orbit lie in relativistic treatments. Through this research direction, I will be able to better model
%  many chemical phenomena, such as charge transfer and reaction dynamics, which are often mediated by spin--forbidden processes.

%----------------------------------------------------------------------------------------
%	EDUCATION SECTION
%----------------------------------------------------------------------------------------

\section{\centerline{EDUCATION}} 

\vspace{8pt} % Gap between title and text

{\sl Doctor of Philosophy (Pursuant)}, Chemistry \hfill (Projected) May 2018 \\ 
University of Washington, Seattle, WA   \\ 
Adviser: Dr. Xiaosong Li
 
{\sl Bachelor of Science (Magna Cum Laude)}, Chemistry, Mathematics \hfill May 2013  \\ 
Indiana University of Pennsylvania, Indiana, PA \\
Adviser: Dr. Jaeju Ko

%----------------------------------------------------------------------------------------
 
\vspace{0.2in} % Some whitespace between sections

%----------------------------------------------------------------------------------------
%	PROFESSIONAL EXPERIENCE SECTION
%----------------------------------------------------------------------------------------

\section{\centerline{PROFESSIONAL EXPERIENCE}} 

\vspace{8pt} % Gap between title and text

{\sl Graduate Research Assistant} \hfill July 2013 -- Present \\
University of Washington, Seattle, WA

{\sl Graduate Teaching Assistant} \hfill September 2013 -- Present \\
University of Washington, Seattle, WA 
\begin{itemize} \itemsep -2pt
  \item Provided supplementary lecture instruction for the undergraduate General Chemistry course series
    as well as full instruction for the associated lab coursework. This job entailed full responsibility
    for two sections of thirty students during a supplemental lecture series referred to as a quiz section.
    I was responsible for the evaluation of students' performance both in lecture coursework (exams, homework assignments)
    as well as the lab (lab reports).
  \item Provided supplementary lecture instruction for the final lecture course (in a series of three) of
    the undergraduate Organic Chemistry course series. Primarily, I was responsible for evaluating students' performance on
    Exams. In addition, I was responsible for making myself available for course related questions through
    structured office hours.
  \item Aided in the development of new course material for the second lecture course (in a series of two),
    of the graduate Quantum Chemistry course series. The developed course material focused on the practical
    development of basic quantum chemical methods, such as Hartree-Fock and Density Functional Theory, using
    the MATLAB development environment. The students developed, from scratch, a working implementation
    of these methods with the aid of provided course material developed by my self and the primary course
    instructor. For this course, I also had the primary responsibility for evaluating students' progress
    through the grading of exams and assigned course work.
\end{itemize}

{\sl Undergraduate Research Assistant} \hfill September 2011 -- May 2013 \\
Indiana University of Pennsylvania, Indiana, PA

{\sl Chemistry Tutor} \hfill August 2010 -- May 2011 \\
Indiana University of Pennsylvania Disability Services, Indiana, PA 
\begin{itemize}
  \item Provided supplemental course instruction for the General and Organic Chemistry
    series for students with disabilities. This primarily entailed meeting with students
    individually on a weekly basis to aid in their understanding of course material and 
    to ensure that they were not falling behind in coursework.
\end{itemize}

{\sl Information Technology Technician} \hfill August 2009 -- January 2010 \\
Central Michigan University Information Technology, Mt. Pleasant, MI
\begin{itemize}
  \item Primarily responsible for solving network connectivity issues for incoming students
    at Central Michigan University.
\end{itemize}

%----------------------------------------------------------------------------------------

\vspace{0.2in} % Some whitespace between sections

%----------------------------------------------------------------------------------------
%	COMPUTER SKILLS SECTION
%----------------------------------------------------------------------------------------

\section{\centerline{COMPUTATIONAL PROFICIENCY}}

\vspace{8pt} % Gap between title and text

I am very proficient in the following computational areas:
\begin{itemize} \itemsep -2pt
  \item {\sl Programming Languages:} C/C++/C\#, FORTRAN 77/95/03, Java
  \item {\sl Scripting Languages:} Python, Julia, R, MATLAB, Octave, C/Bash Shell
  \item {\sl Libraries / Paradigms:} OpenMP, TBB, MPI, OpenGL, CUDA,
  OpenACC, PThreads
  \item {\sl Software:} Gaussian, Mathematica, MATLAB
\end{itemize}

I have contributed to the development of the following software packages:
\begin{itemize} \itemsep -2pt
  \item Chronus Quantum Chemistry (ChronusQ) Software Package ({\sl Principle Developer})
  \item Gaussian ({\sl Contributor})
  \vspace{-6pt}
  \begin{itemize}\itemsep -2pt
    \item Analytical hessians of time--dependent density functional theory
    \item Chebyshev expansion of the quantum propagator in real--time density functional theory
  \end{itemize}
\end{itemize}

%----------------------------------------------------------------------------------------

\vspace{0.2in} % Some whitespace between sections

%----------------------------------------------------------------------------------------
%	PUBLICATIONS SECTION
%----------------------------------------------------------------------------------------

\section{\centerline{PUBLICATIONS}} 

\vspace{15pt} % Gap between title and text
\begin{etaremune}
  \item Petrone,~A.; \me; Lingerfelt,~D.~B.; \xsli;
        ``\emph{Ab Initio Transient Raman Analysis}"
	  \emph{J. Phys. Chem. A.}, \textbf{2017}, Submitted.
  \item Petrone,~A.; Lingerfelt,~D.~B.; \me; \xsli;
        ``\emph{Ab Initio Transient Vibrational Spectral Analysis}"
	  \emph{J. Phys. Chem. Lett.}, \textbf{2016}, 7, 4501--4508.
  \item \me; Goings,~J.; \xsli;
	``\emph{Accelerating Real--Time Time-Dependent Density Functional Theory 
	        with a Non--Recursive Chebyshev Expansion of the Quantum 
                Propagator}"
	  \emph{J. Chem. Theor. Comp.}, \textbf{2016}, 12(11) 5333--5338.
  \item \me; Egidi,~F.; \xsli;
	``\emph{Relativistic Two-Component Particle-Particle Tamm--Dancoff 
	        Approximation}"
	  \emph{J. Chem. Theor. Comp.}, \textbf{2016}, 12(11), 5379--5384.
  \item Lingerfelt,~D.~B.; \me; Petrone,~A; \xsli; 
        ``\emph{Direct ab Initio (Meta-)Surface-Hopping Dynamics}", 
        \emph{J. Chem. Theor. Comp.}, \textbf{2016}, 12(3), 935--945.
\end{etaremune}
%  Papers that are on hold for now:
%
%  \item \me*; Lingerfelt, D.~B.*; Petrone, A.; \xsli;
%	``\emph{Ab Initio Surface Hopping Dynamics within the Particle-Particle 
%	        Tamm--Dancoff Approximation}"
%          \emph{J. Chem. Phys. Comm.}, \textbf{2016}, \emph{Submitted}.
%\begin{center}
%*Authors contributed equally to this work
%\end{center}

\vspace{0.2in} % Some whitespace between sections

\section{\centerline{CURRENT SOFTWARE CITATIONS}} 
\vspace{15pt} % Gap between title and text

\begin{etaremune}
  \item \xsli; Valeev,~E.~F.; \me; Ding,~F.; Liu,~H.; Goings,~J.~J.; Petrone,~A.; Lestrange,~P.;
        \emph{Chronus Quantum, Beta Version}, \url{http://www.chronusquantum.org}, \textbf{2016}.


\item Frisch,~M.~J.; Trucks,~G.~W.; Schlegel,~H.~B.; Scuseria,~G.~E.; Robb,~M.~A.;
      Cheeseman,~J.~R.; Scalmani,~G.; Barone,~V.; Petersson,~G.~A.; Nakatsuji,~H.; \xsli;
      Caricato,~M.; Marenich,~A.; Bloino,~J.; Janesko,~B.~G.; Gomperts,~R.; Mennucci,~B.;
      Hratchian,~H.~P.; Izmaylov,~A.~F.; Sonnenberg,~J.~L.; \me; Ding,~F.; Lipparini,~F.;
      Egidi,~F.; Goings,~J.; Peng,~B.; Petrone,~A.; Ortiz,~J.~V.; Zakrzewski,~V.~G.; Gao,~J.;
      Rega,~N.; Zheng,~G.; Liang,~W.; Hada,~M.; Ehara,~M.; Toyota,~K.; Fukuda,~R.; Hasegawa,~J.;
      Ishida,~M.; Nakajima,~T.; Honda,~Y.; Kitao,~O.; Nakai,~H.; Vreven,~T.; Throssell,~K.; Montgomery~Jr.,~J.~A.;
      Peralta,~J.~E.; Ogliaro,~F.; Bearpark,~M.; Heyd,~J.~J.; Brothers,~E.; Kudin,~K.~N.; Staroverov,~V.~N.; Keith,~T.;
      Kobayashi,~R.; Normand,~J.; Raghavachari,~K.; Rendell,~A.; Burant,~J.~C.; Iyengar,~S.~S.;
      Tomasi,~J.; Cossi,~M.; Millam,~J.~M.; Klene,~M.; Adamo,~C.; Cammi,~R.;
      Ochterski,~J.~W.; Martin,~R.~L.; Morokuma,~K.; Farkas,~O.; Foresman,~J.~B.; and Fox~,~D.~J.;
      \emph{Gaussian Development Version, Revision I.09},
      Gaussian, Inc., Wallingford CT, \textbf{2016}.

\item Frisch,~M.~J.; Trucks,~G.~W.; Schlegel,~H.~B.; Scuseria,~G.~E.; Robb,~M.~A.;
      Cheeseman,~J.~R.; Scalmani,~G.; Barone,~V.; Petersson,~G.~A.; Nakatsuji,~H.; \xsli;
      Caricato,~M.; Marenich,~A.; Bloino,~J.; Janesko,~B.~G.; Gomperts,~R.; Mennucci,~B.;
      Hratchian,~H.~P.; Izmaylov,~A.~F.; Sonnenberg,~J.~L.; \me; Ding,~F.; Lipparini,~F.;
      Egidi,~F.; Goings,~J.; Peng,~B.; Petrone,~A.; Ortiz,~J.~V.; Zakrzewski,~V.~G.; Gao,~J.;
      Rega,~N.; Zheng,~G.; Liang,~W.; Hada,~M.; Ehara,~M.; Toyota,~K.; Fukuda,~R.; Hasegawa,~J.;
      Ishida,~M.; Nakajima,~T.; Honda,~Y.; Kitao,~O.; Nakai,~H.; Vreven,~T.; Throssell,~K.; Montgomery~Jr.,~J.~A.;
      Peralta,~J.~E.; Ogliaro,~F.; Bearpark,~M.; Heyd,~J.~J.; Brothers,~E.; Kudin,~K.~N.; Staroverov,~V.~N.; Keith,~T.;
      Kobayashi,~R.; Normand,~J.; Raghavachari,~K.; Rendell,~A.; Burant,~J.~C.; Iyengar,~S.~S.;
      Tomasi,~J.; Cossi,~M.; Millam,~J.~M.; Klene,~M.; Adamo,~C.; Cammi,~R.;
      Ochterski,~J.~W.; Martin,~R.~L.; Morokuma,~K.; Farkas,~O.; Foresman,~J.~B.; and Fox~,~D.~J.;
      \emph{Gaussian 16, A.03},
      Gaussian, Inc., Wallingford CT, \textbf{2016}.
\end{etaremune}

\vspace{0.2in} % Some whitespace between sections
%----------------------------------------------------------------------------------------
%	HONORS SECTION
%----------------------------------------------------------------------------------------

\section{\centerline{HONORS}} 

\vspace{-5pt} % Reduce space between section title and contents

\begin{center}
Nicole A. Boand Endowed Fellowship in Chemistry \hfill ARCS Foundation (2016) \\
Early Bird Research Assistantship (EBRA) \hfill University of Washington (2013) \\
Excellence in Chemistry Graduate Fellowship Award (ECGFA) \hfill University of Washington (2013)\\
Provost Scholar \hfill Indiana University of Pennsylvania (2013)
\end{center}

%----------------------------------------------------------------------------------------

\vspace{0.2in} % Some whitespace between sections

\section{\centerline{PRESENTATIONS}} 

\vspace{15pt} % Gap between title and text
\begin{etaremune}
  \item \me; Goings, J. J.; \xsli;
        ``\emph{Accelerating Real--Time Time--Dependent Density Functional Theory with a Chebyshev 
	  Expansion of the Quantum Propagator}";
	Theory and Applications of Computational Chemistry (TACC) 2016, Seattle, WA.
	\textbf{2016}. Poster.
  \item Egidi F.; \me; \xsli;
        ``\emph{Electronic Structure Methods for Relativistic Effects in Excited States}";
	Low Scaling and Unconventional Electronic Structure Theory (LUEST) 2016, Telluride, CO.
	\textbf{2016}. Poster.
  \item \me; Yang, W.; \xsli;
        ``\emph{Moving past the particle-hole description of excited states: Affordable methodologies}";
	Chemical Congress of Pacific Basin Societies (PacifiChem) 2015, Honolulu, HI.
	\textbf{2015}. 
	PHYS: Recent Progress in Molecular Theory for Excited-state Electronic Structure and
	Dynamics. Oral Presentation.
%  \item Scalmani, G.; Frisch, M.; \me; \xsli;
%	``\emph{On the analytical second derivatives of the excited state energy in the 
%	        framework of Time-Dependant Density Functional Theory}";
%	248h American Chemical Society National Meeting \& Exposition, San Francisco, CA.
%	\textbf{2014}. COMP: Quantum Chemistry Symposium. Oral Presentation.
  \item \me; \ko; Ondrechen, M. J.;
	``\emph{Computational approach to the prediction of enzyme specificities}";
	245th American Chemical Society National Meeting \& Exposition, New Orleans, LA.
	\textbf{2013}. Sci-Mix Poster Session. Poster.
  \item \me; \ko; Ondrechen, M. J.;
	``\emph{Computational approach to the prediction of enzyme specificities}";
	245th American Chemical Society National Meeting \& Exposition, New Orleans, LA.
	\textbf{2013}. Division of Computers in Chemistry Poster Session. Poster.
  \item \me; \ko;
	``\emph{Prediction of Relative Activities of Enzymes using Computed Chemical Properties}";
	American Chemical Society Student Member Symposium, Duquesne University. Duquesne, PA.
	\textbf{2012}. Poster.
  \item \me; \ko;
	``\emph{Prediction of Relative Activities of Enzymes using Computed Chemical Properties}";
	Undergraduate Research Forum, Indiana University of Pennsylvania. Indiana, PA.
	\textbf{2012}. Poster.
  \item Ford, J.; \ko; Mintmier, B.; Machovia, T.; Kang, M.; Owens, A.; \me;
	``\emph{Undergraduate Research in Biomass Utilization Symposium}",
	Pennsylvania Association of the Council of Trustees Conference, Indiana University of Pennsylvania.
	Indiana, PA. \textbf{2011}. Oral Presentation.
\end{etaremune}
%----------------------------------------------------------------------------------------

\vspace{0.2in} % Some whitespace between sections

%----------------------------------------------------------------------------------------
%	MEMBERSHIPS SECTION
%----------------------------------------------------------------------------------------

\section{\centerline{MEMBERSHIPS}} 

\vspace{-5pt} % Reduce space between section title and contents

\begin{center}
Alpha Chi Sigma ($\mathrm{AX}\Sigma$), $\Gamma\mathrm{T}$ Chapter, Professional Chemistry Fraternity\\
American Chemical Society (ACS), Computers in Chemistry Division
\end{center}

%----------------------------------------------------------------------------------------

\vspace{0.2in} % Some whitespace between sections

%----------------------------------------------------------------------------------------
%	INTERESTS SECTION
%----------------------------------------------------------------------------------------

\section{\centerline{INTERESTS}} 

\vspace{-5pt} % Reduce space between section title and contents

\begin{center}
World percussion, kayaking, hiking, climbing, Linux kernel development
\end{center} 

%----------------------------------------------------------------------------------------

\end{resume} 
\end{document}
