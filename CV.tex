%%%%%%%%%%%%%%%%%%%%%%%%%%%%%%%%%%%%%%%%%
% Long Professional Curriculum Vitae
% LaTeX Template
% Version 1.1 (9/12/12)
%
% This template has been downloaded from:
% http://www.latextemplates.com
% Original author:
% Rensselaer Polytechnic Institute (http://www.rpi.edu/dept/arc/training/latex/resumes/)
%
% Important note:
% This template requires the res.cls file to be in the same directory as the
% .tex file. The res.cls file provides the resume style used for structuring the
% document.
%
%%%%%%%%%%%%%%%%%%%%%%%%%%%%%%%%%%%%%%%%%

%----------------------------------------------------------------------------------------
%	PACKAGES AND OTHER DOCUMENT CONFIGURATIONS
%----------------------------------------------------------------------------------------

\let\latexnofiles\nofiles
\let\nofiles\relax
\documentclass[10pt]{res} % Use the res.cls style, the font size can be changed to 11pt or 12pt here

\usepackage{helvet} % Default font is the helvetica postscript font
%\usepackage{newcent} % To change the default font to the new century schoolbook postscript font uncomment this line and comment the one above
\usepackage{etaremune}
\usepackage{hyperref}

\newsectionwidth{0pt} % Stops section indenting

% Name shortcuts
\newcommand*\me[0]{{\bf Williams--Young,~D.~B.}}
\newcommand*\xsli[0]{Li,~X.}
\newcommand*\cy[0]{Yang,~C.}
\newcommand*\bdj[0]{de~Jong,~W.A.}
\newcommand*\ko[0]{Ko,~J.}

%\newcommand{\invited}{\item[$*$\theenumi.]}
\newcommand{\invited}{\refstepcounter{enumi}\item[$*$\theenumi.]}
\newcommand{\presentation}[6]{%
{#1}; ``\emph{#2}"; {#3}; {#4}; \textbf{#5}; {#6}.%
}

\begin{document}

%----------------------------------------------------------------------------------------
%	YOUR NAME AND ADDRESS(ES) SECTION
%----------------------------------------------------------------------------------------

\name{Dr. David Williams--Young, Ph.D\\ \\} % Your name at the top

% If you don't want one of the addresses, simply remove all the text in the first or second \address{} bracket

\address{Principal Quantum Software Engineer\\
Microsoft Azure Quantum Elements\\
Microsoft Corporation\\
%50F-1645 (office),  
%MS 50F-1650 (mail)\\
One Microsoft Way, 
Redmond, WA 98052\\
%+1 (510) 495-2189\\
%dbwy@lbl.gov%
} % Your address 1

\address{ }

%----------------------------------------------------------------------------------------

\begin{resume}

%%----------------------------------------------------------------------------------------
%%	OBJECTIVE SECTION
%%----------------------------------------------------------------------------------------
%
%\section{\centerline{OBJECTIVE}}
%
%\vspace{8pt} % Gap between title and text
%
%To obtain a research position relating to high--performance scientific computing and numerical
%linear algebra and their interplay with methods development in electronic structure theory.


%----------------------------------------------------------------------------------------
%       RESEARCH INTERESTS
%----------------------------------------------------------------------------------------

\section{\centerline{PROFESSIONAL INTERESTS}}
\begin{itemize} \itemsep -2pt
%  \item Development of low--scaling relativistic electronic structure methods to describe
%  strongly correlated systems, such as transition metal complexes commonly found in
%  the vicinity of the active sites of enzymes.
%  \item Application of abstract mathematical paradigms, such as algebraic topology, to
%  develop elegant and novel solutions to the problems that arise in ab initio electronic
%  structure theory.
%  \item Development of high--performance algorithms to treat the electronically non--adiabatic
%  dynamics of quantum molecular systems.
  \item Development of high--performance %and reduced scaling 
  electronic structure methods on emerging architectures. %(GPUs, FPGAs, etc).
%  \item Development of novel field-theoretic methods for the elucidation of relativistic
%  electronic structure in superheavy elements.
%  \item Development of novel quantum algorithms for the simulation of the relativistic
%  many-body problem.
  \item Development and microarchitecture optimization of high--performance linear algebra software.
  \item Leverage of modern C++ to develop clean, resuable, and extensible scientific software.
  \item The intersection of high-performance computing and quantum information science.
  %\item Application of algebraic topology and differential geometry to develop elegant and 
  %novel solutions to the problems that arise in ab initio electronic structure theory.
\end{itemize}

%\section{\centerline{RESEARCH SUMMARY}}
%\begin{itemize} \itemsep -2pt
%  \item \textbf{Relativistic Excited State Electronic Structure Theory}, \emph{2014 -- Present}. 
%  A primary focus of my current and past research has been the development of efficient relativistic
%  excited state methods for electronic structure theory. Recently, I have extended the
%  particle--particle Tamm--Dancoff approximation, a method well known in \emph{ab initio}
%  nuclear structure theory, to two--component relativistic Hamiltonians. Using this method,
%  I was able to describe, with excellent accuracy, the fine--structure splitting of the excited
%  states of a set of atomic and molecular systems [\emph{J. Chem. Theor. Comp.}, \textbf{2016}, 12(11), 5379--5384].
%  I am also currently developing relativistic extensions for various other post-SCF methods such as
%  Coupled Cluster and Configuration Interaction.\\
%
%  \item \textbf{Molecular Dynamics}, \emph{2015 -- Present}. Recently, much of my research effort
%  has been directed to the development of novel and efficient electronic structure methods to describe
%  the electronic and nuclear dynamics of molecular systems in an \emph{ab initio} manner. I have developed
%  a revised algorithm for the evaluation of the quantum propagator of time--dependent density functional
%  theory (TDDFT) through the Chebyshev expansion of the matrix exponential [\emph{J. Chem. Theor. Comp.}, \textbf{2016}, 12(11) 5333--5338].
%  In principle, this development makes it possible to achieve linear--scaling electronic dynamics simulations
%  through the use of sparse matrix manipulation.\\
%  \\
%  Further, in collaboration with my colleagues in the Li research group, I have been involved with several 
%  exciting developments in the field of nonadiabatic excited state nuclear dynamics. The key to the usability
%  of any nonadiabatic molecular dynamics method is the underlying efficiency of the algorithms used to obtain
%  the required physical moieties, such as gradients and nonadiabatic couplings, throughout the simulation.
%  Much of my work in this field has been in the development of extremely efficient and scalable methods to
%  obtain these quantities. My work has facilitated the development of several methodological developments in the
%  field, including direct \emph{ab initio} rare event sampling through the meta--surface hopping method
%  [\emph{J. Chem. Theor. Comp.}, \textbf{2016}, 12(3), 935--945], and the analysis of transient vibrational signatures
%  to describe confomational changes through conformational changes 
%  [\emph{J. Chem. Phys. Comm.}, \textbf{2016}, 7, 4501--4508].
%\end{itemize}
%
%\section{\centerline{FUTURE RESEARCH STATEMENT}}
%  My plan for future research directions involves the combination of my previous research topics through
%  the application of relativistic hamiltonians to describe the nonadibatic behavior of molecular dynamics. To properly
%  treat nonadiabatic transitions between electronic states of differing spin multiplicities, such as intersystem crossings
%  and phosphorescence, a proper treatment of spin--orbit coupling is of utmost importance. While several approaches have
%  been to describe spin--orbit perturbatively in the non--relativistic regime, the only method to properly describe 
%  the effects of spin--orbit lie in relativistic treatments. Through this research direction, I will be able to better model
%  many chemical phenomena, such as charge transfer and reaction dynamics, which are often mediated by spin--forbidden processes.

\vspace{0.2in} % Some whitespace between sections

%----------------------------------------------------------------------------------------
%	PROFESSIONAL EXPERIENCE SECTION
%----------------------------------------------------------------------------------------

\section{\centerline{PROFESSIONAL EXPERIENCE}} 

\vspace{8pt} % Gap between title and text

{\sl \bf Principal Quantum Software Engineer} \dotfill August 2024 -- Present \\
Microsoft Azure Quantum Elements \\
Microsoft Corporation, Redmond, WA

{\sl \bf Research Scientist (Career)} \dotfill March 2023 -- August 2024 \\
{\sl \bf Research Scientist (Career-Track)} \dotfill February 2021 -- March 2023 \\
{\sl \bf Postdoctoral Fellow} \dotfill July 2018 -- February 2021\\
Applied Mathematics and Computational Research Division \\
Lawrence Berkeley National Laboratory, 
Berkeley, CA

{\sl \bf Graduate Research Assistant} \dotfill July 2013 -- July 2018 \\
{\sl \bf Graduate Teaching Assistant} \dotfill September 2013 -- July 2018 \\
Department of Chemistry \\
University of Washington, Seattle, WA 
%\begin{itemize} \itemsep -2pt
%  \item Provided supplementary lecture instruction for the undergraduate General Chemistry course series
%    as well as full instruction for the associated lab coursework. This job entailed full responsibility
%    for two sections of thirty students during a supplemental lecture series referred to as a quiz section.
%    I was responsible for the evaluation of students' performance both in lecture coursework (exams, homework assignments)
%    as well as the lab (lab reports).
%  \item Provided supplementary lecture instruction for the final lecture course (in a series of three) of
%    the undergraduate Organic Chemistry course series. Primarily, I was responsible for evaluating students' performance on
%    Exams. In addition, I was responsible for making myself available for course related questions through
%    structured office hours.
%  \item Aided in the development of new course material for the second lecture course (in a series of two),
%    of the graduate Quantum Chemistry course series. The developed course material focused on the practical
%    development of basic quantum chemical methods, such as Hartree-Fock and Density Functional Theory, using
%    the MATLAB development environment. The students developed, from scratch, a working implementation
%    of these methods with the aid of provided course material developed by my self and the primary course
%    instructor. For this course, I also had the primary responsibility for evaluating students' progress
%    through the grading of exams and assigned course work.
%\end{itemize}

%{\sl \bf Undergraduate Research Assistant} \dotfill September 2011 -- May 2013 \\
%Department of Chemistry \\
%Indiana University of Pennsylvania, Indiana, PA

%{\sl Chemistry Tutor} \hfill August 2010 -- May 2011 \\
%Indiana University of Pennsylvania Disability Services, Indiana, PA 
%\begin{itemize}
%  \item Provided supplemental course instruction for the General and Organic Chemistry
%    series for students with disabilities. This primarily entailed meeting with students
%    individually on a weekly basis to aid in their understanding of course material and 
%    to ensure that they were not falling behind in coursework.
%\end{itemize}

%{\sl Information Technology Technician} \hfill August 2009 -- January 2010 \\
%Central Michigan University Information Technology, Mt. Pleasant, MI
%%\begin{itemize}
%%  \item Primarily responsible for solving network connectivity issues for incoming students
%%    at Central Michigan University.
%%\end{itemize}

%----------------------------------------------------------------------------------------

\vspace{0.2in} % Some whitespace between sections

%----------------------------------------------------------------------------------------
%	EDUCATION SECTION
%----------------------------------------------------------------------------------------
\section{\centerline{EDUCATION}} 

\vspace{8pt} % Gap between title and text

{\sl\bf  Doctor of Philosophy} (Ph.D.), Chemistry \dotfill May 2018 \\ 
University of Washington, Seattle, WA   \\ 
Adviser: Dr. Xiaosong Li \\
Dissertation: \emph{Towards Efficient and Scalable Electronic Structure Methods for 
the Treatment of Relativistic Effects and Molecular Response}
 
{\sl\bf Bachelor of Science} (B.S., \emph{Magna Cum Laude}), Chemistry, Mathematics \dotfill May 2013  \\ 
Indiana University of Pennsylvania, Indiana, PA \\
Adviser: Dr. Jaeju Ko

%----------------------------------------------------------------------------------------
 
\vspace{0.2in} % Some whitespace between sections

%----------------------------------------------------------------------------------------
%	COMPUTER SKILLS SECTION
%----------------------------------------------------------------------------------------

\newpage
\section{\centerline{COMPUTATIONAL PROFICIENCY}}

\vspace{8pt} % Gap between title and text

I am a specialist in the following computational areas:
\begin{itemize} \itemsep -2pt
  \item {\sl Programming Languages:} C++, CUDA
  \item {\sl Libraries / Paradigms:} OpenMP, MPI, BLAS/(Sca)LAPACK
\end{itemize}

I am very proficient in the following computational areas:
\begin{itemize} \itemsep -2pt
  \item {\sl Programming Languages:} C, FORTRAN 77/95/03, HIP, SYCL
  \item {\sl Scripting Languages:} Python, Julia, Octave, C/Bash Shell
  \item {\sl Libraries / Paradigms:} TBB, OpenACC
  \item {\sl Software:} Mathematica, MATLAB
\end{itemize}

I am the primary developer of the following open-source software packages:
\begin{itemize} \itemsep -2pt
  \item {\bf GauXC}:   Enabling high-performance density functional theory calculations on exascale architectures
  \item {\bf MACIS}:   Massively parallel selected configuration interaction methods
  \item {\bf ExchCXX}: GPU-accelerated library for the evaluation of exchange-correlation functionals
\end{itemize}

I have contributed to the development of the following software packages:
\begin{itemize} \itemsep -2pt
  \item The Chronus Quantum (ChronusQ) Software Package
  \item NWChemEx
  \item Massively Parallel Quantum Chemistry (MPQC4)
  \item TiledArray
  \item Gaussian
\end{itemize}

%----------------------------------------------------------------------------------------

\vspace{0.2in} % Some whitespace between sections

%----------------------------------------------------------------------------------------
%	PUBLICATIONS SECTION
%----------------------------------------------------------------------------------------

\section{\centerline{PUBLICATIONS}} 
\vspace{-0.1in} % Some whitespace between sections
\begin{center}
{\bf 44 in press publications, %
%1 accepted pending publication, 
1 submitted manuscript, \\
  9 as first author (named), 3 as equally contributing (first) author, 6 invited.}\\

* Indicates equal contribution to published work. \\$\dagger$ Indicates that the publication was invited.
\end{center}

\subsection{Lawrence Berkeley National Laboratory\hfill 2018-2024}

\begin{etaremune}
  \item Shen, Y.; Camps, D.; Szaz, A.; Darbha, S.; Klymko, K.; \me; Tubman, N.B.; Van Beeumen, R.;
        ``\emph{Estimating Eigenenergies from Quantum Dynamics: A Unified Noise-Resilient Measurement-Driven Approach}"
        \textbf{2023}. Submitted.
  \item Hirsbrunner, M.R.; Mullinax, J.W.; Shen, Y.; \me; Klymko, K.; Van Beeumen, R.; Tubman, N.M.;
        ``\emph{A circuit-generated quantum subspace algorithm for the variational quantum eigensolver}"
       \emph{J. Chem. Phys.} \textbf{2024}. 161, 164103.
  \item Bylaska, E.; Panyala, A.; Bauman, N.; Peng, B.; Pathak, H.; Mejia-Rodriguez, D.; Govind, N.; \me; Apra, E.; 
        Bagusetty, A.; Mutlu, E.; Jackson, K.; Baruah, T.; Yamamoto, Y.; Pederson, M.; Withanage, K.; Pedroza-Montero, J.; 
        Bilbrey, J.; Choudhury, S.; Firoz, J.; Herman, K.; Xantheas, S.; Rigor, P.; Vila, F.; Rehr, J.; Fung, M.; Grofe, A.; 
        Johnston, C.; Baker, N.; Kaneko, K.; Liu, H.; Kowalski, K.;
        ``\emph{Electronic structure simulations in the cloud computing environment}"
        \emph{J. Chem. Phys.} \textbf{2024}. 161, 150902.
  \item Kovtun, M.; Lambros, E.; Liu, A.; Tang, D.; \me; \xsli;
        ``\emph{Accelerating Relativistic Exact-Two-Component Density Functional Theory Calculations with Graphical Processing Units}"
        \emph{J. Chem. Theory Comput.} \textbf{2024}. 20, 18, 7694-7699
  \item $\dagger$ Burton, H.; Dong, S.; Ghosh, S.; Gu, B.; Jackson, N.; Keefer, D.; Lu, Y.; Monroe, J.; Peng, B.; Pieri, E.; 
        Spackman, P.; Vacher, M.; Vuckovic, S.; \me; Yang, Z.; Yue, S.; Zerze, G.; Zhu, T.;
        ``\emph{Editorial: JCTC Early Career Board Selects}"
        \emph{J. Chem. Theory Comput.}; \textbf{2024}. 
  \item Poole, D.; \me; Jiang, A.; Glick, Z.; Sherrill, C.D..;
        ``\emph{A modular, composite framework for the utilization of reduced-scaling Coulomb and Exchange construction algorithms: Design and implementation}";
        \emph{J. Chem. Phys.}; \textbf{2024}. 161, 052503.
  \item Alvertis, A.M.; \me; Bruneval, F; Neaton, J.B.;
       ``\emph{Influence of Electronic Correlations on Electron–Phonon Interactions of Molecular Systems with the GW and Coupled Cluster Methods}"
       \emph{J. Chem. Theory Comput.}; \textbf{2024}. 20, 14, 6175–6183. 
  \item $\dagger$ Blum, V.; Asahi, R.; Autschbach, J.; Bannwarth, C.; Bihlmayer, G.; Blügel, S.; Burns, L. A.; 
        Crawford, T. D.; Dawson, W.; de Jong, W. A.; Draxl, C.; Filippi, C.; Genovese, L.; Giannozzi, P.; 
        Govind, N.; Hammes-Schiffer, S.; Hammond, J. R.; Hourahine, B.; Jain, A.; Kanai, Y.; Kent, P. R. C.; 
        Larsen, A. H.; Lehtola, S.; Li, X.; Lindh, R.; Maeda, S.; Makri, N.; Moussa, J.; Nakajima, T.; 
        Nash, J. A.; Oliveira, M. J. T.; Patel, P. D.; Pizzi, G.; Pourtois, G.; Pritchard, B. P.; Rabani, E.; 
        Reiher, M.; Reining, L.; Ren, X.; Rossi, M.; Schlegel, H. B.; Seriani, N.; Slipchenko, L. V.; Thom, A.; 
        Valeev, E. F.; Van Troeye, B.; Visscher, L.; Vlcek, V.; Werner, H.-J.; \me; Windus, T.
        ``\emph{Roadmap on methods and software for electronic structure based simulations in chemistry and materials}"
        \emph{Electronic Structure}; \textbf{2024}. 
  \item \me; Yuwono, S.; DePrince III, A.E.; \cy; 
        ``\emph{Approximate Exponential Integrators for Time-Dependent Equation-of-Motion Coupled Cluster Theory}”
	\emph{J. Chem. Theory Comput.}; \textbf{2023}. 19, 24, 9177–9186.
  \item Di Felice, R.; Mayes, M.; Richard, R.; \me; Chan, G.K.L; \bdj; 
        Govind, N.; Head-Gordon, M.; Hermes, M.; Kowalski, K.; Li, X.; Lischka, H.; Mueller, K.; 
	Mutlu, E.; Niklasson, A.; Pederson, M.; Peng, B.; Shepard, R.; Valeev, E.; van Schilfgaarde, M.; 
	Vlaisavljevich, B.; Windus, T.; Xantheas, S.; Zhang, X.; Zimmerman, P.;
	``\emph{A Perspective on Sustainable Computational Chemistry Software Development and Integration}"
	\emph{J. Chem. Theory Comput.}; \textbf{2023}. 19, 20, 7056–7076.
  \item Ko, T.; Heindel, J.; Guan, X.;  Head-Gordon, T.; \me; \cy; 
        ``\emph{Using Diffusion Maps to Analyze Reaction Dynamics for a Hydrogen Combustion Benchmark Dataset}"
        \emph{J. Chem. Theory Comput.}; \textbf{2023}. 19, 17, 5872–5885.
  \item $\dagger$ \me; Asadchev, A.; Popovici, D.T; Clark, D.; Waldrop, J.; Windus, T.L.;
        Valeev, E.F.; \bdj;
        ``\emph{Distributed Memory, GPU Accelerated Fock Construction for Hybrid, Gaussian 
                Basis Density Functional Theory}";
        \emph{J. Chem. Phys.}; \textbf{2023}. 158, 234104.
  \item $\dagger$ \me; Tubman, N.M.; Mejuto-Zaera, C.; \bdj;
        ``\emph{A Parallel, Distributed Memory Implementation of the Adaptive 
                Sampling Configuration Interaction Method}";
        \emph{J. Chem. Phys.}; \textbf{2023}. 158, 214109. 
  \item Richard, R.; Keipert, K.; Waldrop, J.; Keçeli, M.; \me; Bair, R.; Boschen, J.; 
        Crandall, Z.; Gasperich, K.; Mahmud, Q.; Panyala, A.; Valeev, E.F.; van Dam, H,; 
        \bdj; Windus, T.;
       ``\emph{PluginPlay: Enabling Exascale Scientific Software One Module at a Time}";
        \emph{J. Chem. Phys.}; \textbf{2023}. 158, 184801. 
  \item Shen, Y.; Klymko, K.; Sud, J.; \me; \bdj; Tubman, N.M;
    ``\emph{Real-Time Krylov Theory for Quantum Computing Algorithms}";
    \textbf{2023}. \emph{Quantum}. 7, 1066.
  \item Gomes, N.; \me;  \bdj;
    ``\emph{Computing the Many-Body Green's Function with Adaptive Variational 
            Quantum Dynamics}";
    \emph{J. Chem. Theory Comput.}; \textbf{2023}. 19, 11, 3313–3323.
  \item Sid-Lakhdar, W.M.; Cayrols, S.; Bielich, D.; Abdelfattah, A.; Luszczek, P.  Gates, M.; Tomov, S.; 
        Johansen, H.; \me; Davis, T.; Dongerra, J.; Anzt, H.; 
        ``\emph{PAQR: Pivoting Avoiding QR factorization}"; 
        2023 IEEE International Parallel and Distributed Processing Symposium (IPDPS), St. Petersburg, FL, USA; \textbf{2023}, pp. 322-332.
  \item Mejuto-Zaera, C.; Tzeli, D; \me; Tubman, N.M.; Matoušek, M.; Brabec, J.; Veis, L.; Xantheas, S; \bdj; 
       ``\emph{The Effect of Geometry, Spin and Orbital Optimization in Achieving Accurate, Fully-Correlated Results for Iron-Sulfur 
                Cubanes}"; \emph{J. Chem. Theory Comput}; \textbf{2022}. 18(2), 687–702.
  \item Bez, J.L.; Tang, H.; Xie, B.; \me; Latham, R.; Ross, R.; Oral, S.; Byna, S.;
        ``\emph{I/O Bottleneck Detection and Tuning:Connecting the Dots using Interactive Log Analysis}";
        \emph{2021 IEEE/ACM 6th International Parallel Data Systems Workshop (PDSW)}, 
        \textbf{2021}.
  \item \me; Bagusetty, A.; \bdj; Doerfler, D.; vam Dam, H.J.J.; Vazquez--Mayagoitia, A.;
        Windus, T.L.; Yang, C.;
        ``\emph{Achieving Performance Portability in Gaussian Basis Set Density Functional Theory 
                on Accelerator Based Architectures}"; \emph{Parallel Computing},
        \textbf{2021}, 108, 102829.
  \item Ahmed, H.; \me; Ibrahim, K.Z.; \cy;
        ``\emph{Performance Modeling and Tuning for DFT Calculations on Heterogeneous Architectures}";
        \emph{22nd IEEE International Workshop on Parallel and Distributed Scientific and Engineering Computing (PDSEC 2021)},
        \textbf{2021}, pp. 714--722.
  \item Kowalski, K.; Bair, R.; Bauman, N.P.; Boschen, J.S.; Bylaska, E.J.; Daily, J.; \bdj; 
        Dunning, T.; Govind, N.; Harrison, R.J.; Keceli, M.; Keipert, K.; Krishnamoorthy, S.; Kumar, S.;
        Mutlu, E.; Palmer, B.; Panyala, A.; Peng, B.; Richard, R.M.; Straatsma, T.P.; Sushko, P.; Valeev, E.F.;
        Valiev, M.; van Dam, H.J.J.; Waldrop, J.M.; \me; \cy; Zalewski, M.; Windus, T.L.;
       ``\emph{From NWChem to NWChemEx: Evolving with the Computational Chemistry Landscape}"; \emph{Chemical Reviews},
       \textbf{2021}, 121(8), 4962--4998.
  \item \cy; Brabec, J.; Veis, L.; \me; Kowalski, K.;
        ``\emph{Solving Coupled Cluster Equations by the Newton Krylov Method}";
        \emph{Frontiers in Chemistry}, \textbf{2020}, 8:590184.
  \item $\dagger$ \me; \bdj; van Dam, H.J.J.; \cy;
        ``\emph{On the Efficient Evaluation of the Exchange Correlation Potential on Graphics Processing Unit Clusters}";
        \emph{Frontiers in Chemistry}, \textbf{2020}, 8:581058.
  \item \me; Beckman, P.G.; \cy;
        ``\emph{A Shift Selection Strategy for Parallel Shift-Invert Spectrum Slicing in Symmetric Self-Consistent 
        Eigenvalue Computation}";
        \emph{ACM Trans. Math. Soft.}, \textbf{2020}, 46, 4, Article 35 (September 2020).
  \item \me; \cy;
        ``\emph{Parallel Shift-Invert Spectrum Slicing on Distributed Architectures with GPU Accelerators}"
        in \emph{Proceedings of the 49th International Conference on Parallel Processing (ICPP'20)}, \textbf{2020}.
  \item Peng, B.; Van Beeumen, R.; \me; Kowalski, K.; \cy;
        ``\emph{Approximate Green’s Function Coupled Cluster Method Employing 
          Effective Dimension Reduction}";
        \emph{J. Chem. Theor. Comp.}, \textbf{2019}, 15(5), 3185--3196.
\end{etaremune}

\subsection{University of Washington\hfill 2013-2018}

\begin{etaremune}
  \item Koulias, L.N.; \me; Nascimento, D.R.; DePrince, A.E.; \xsli;
        ``\emph{Relativistic Real-Time Time-Dependent Equation-of-Motion Coupled-Cluster}"
        \emph{J. Chem. Theor. Comp.}, \textbf{2019}, 15(12), 6617--6624.
  \item Sun, S.; Beck, R.; \me; \xsli;
        ``\emph{Simulating Magnetic Circular Dichroism Spectra with Real-Time Time-Dependent
          Density Functional Theory in Gauge Including Atomic Orbitals}"
        \emph{J. Chem. Theor. Comp.}, \textbf{2019}, 15(12), 6824--6831.
  \item $\dagger$ \me; Petrone, A.; Sun, S.; Stetina, T.F.; Lestrange, P.; Hoyer, C.E.; 
        Nascimento, D.R.; Koulias, L.; Wildman, A.; Kasper, J.; Goings, J.J.; 
        Ding, F.; DePrince, A.E.; Valeev, E.F.; \xsli;
        ``\emph{The Chronus Quantum (ChronusQ) Software Package}"
        \emph{WIREs Comput. Mol. Sci.} \textbf{2019}, e1436.
  \item Stetina, T.F.; Sun, S.; \me; \xsli;
        ``\emph{Modeling Magneto-Photoabsorption Using Time-Dependent Complex 
          Generalized Hartree-Fock}"
        \emph{ChemPhotoChem}, \textbf{2019}, 3(9), 739--746.
  \item Hoyer, C.; \me; Huang, C.; \xsli;
        ``\emph{Embedding Non-Collinear Two-Component Electronic Structure in 
          a Collinear Quantum Environment}"
        \emph{J. Chem. Phys.}, \textbf{2019}, 150(17), 174114.
  \item Sun, S.; \me; \xsli;
        ``\emph{An Ab Initio Linear Response Method for Computing Magnetic 
          Circular Dichroism Spectra with Non-Perturbative Treatment of 
          Magnetic Field}";
        \emph{J. Chem. Theor. Comp.}, \textbf{2019}, 15(5), 3162--3169.
  \item Sun, S.; \me; Stetina, T.F.; \xsli;
        ``\emph{Generalized Hartree-Fock with a Non-perturbative Treatment 
          of Strong Magnetic Fields: Application to Molecular Spin Phase 
          Transitions}";
        \emph{J. Chem. Theor. Comp.}, \textbf{2019}, 15(1), 348--356.
  \item Petrone, A.*; \me*; Sun, S.; Stetina, T. F.; \xsli;
        ``\emph{An Efficient Implementation of Two-Component Relativistic 
	  Density Functional Theory with Torque-Free Auxiliary Variables}";
	\emph{Eur. Phys. J. B}, \textbf{2018}, 91(7), 169.
  \item Kasper, J.; \me; Vecharynski, E.; \cy; \xsli;
        ``\emph{A Well-Tempered Hybrid Method for Solving Challenging 
	  TDDFT Systems}";
        \emph{J. Chem. Theor. Comp.}, \textbf{2018}, 14(4), 2034--2041.
  \item Lestrange, P.; \me; Jimenez--Hoyos, C.; \xsli;
        ``\emph{An Efficient Implementation of Variation After Projection 
	  Generalized Hartree--Fock}"
        \emph{J. Chem. Theor. Comp.}, \textbf{2018}, 14(2), 588--596.
  \item Barclay, M. S.; Quincy, T. J.; \me; Caricato, M.; Elles, C. G.;
        ``\emph{ Accurate Assignments of Excited-State Resonance Raman 
                 Spectra: A Benchmark Study Combining Experiment and 
                 Theory}"
        \emph{J. Phys. Chem. A.}, \textbf{2017}, 121(41), 7937--7946.
  \item Van Beeuman, R.; \me; Kasper, J.; \cy; Ng, E. G.; \xsli;
        ``\emph{A Model Order Reduction Algorithm for Estimating the 
	        Absorption Spectrum}"
        \emph{J. Chem. Theor. Comp.}, \textbf{2017}, 13(10), 4950--4961.
  \item Egidi, F.*; \me*; Baiardi, A.*; Bloino, J.; Scalmani, G.; Frisch, M.; 
        \xsli; Barone, V.; 
        ``\emph{Effective Inclusion of Mechanical and Electrical Anharmonicity in 
                Excited Electronic States: the VPT2-TDDFT Route}"
          \emph{J. Chem. Theor. Comp.}, \textbf{2017}, 13(6), 2789--2803.
  \item Petrone,~A.*; \me*; Lingerfelt,~D.~B.; \xsli;
        ``\emph{Ab Initio Transient Raman Analysis}"
	  \emph{J. Phys. Chem. A.}, \textbf{2017}, 121(20), 3958--3965.
  \item Petrone,~A.; Lingerfelt,~D.~B.; \me; \xsli;
        ``\emph{Ab Initio Transient Vibrational Spectral Analysis}"
	  \emph{J. Phys. Chem. Lett.}, \textbf{2016}, 7, 4501--4508.
  \item \me; Goings,~J.; \xsli;
	``\emph{Accelerating Real--Time Time-Dependent Density Functional Theory 
	        with a Non--Recursive Chebyshev Expansion of the Quantum 
                Propagator}"
	  \emph{J. Chem. Theor. Comp.}, \textbf{2016}, 12(11) 5333--5338.
  \item \me; Egidi,~F.; \xsli;
	``\emph{Relativistic Two-Component Particle-Particle Tamm--Dancoff 
	        Approximation}"
	  \emph{J. Chem. Theor. Comp.}, \textbf{2016}, 12(11), 5379--5384.
  \item Lingerfelt,~D.~B.; \me; Petrone,~A; \xsli; 
        ``\emph{Direct ab Initio (Meta-)Surface-Hopping Dynamics}", 
        \emph{J. Chem. Theor. Comp.}, \textbf{2016}, 12(3), 935--945.
\end{etaremune}
%  Papers that are on hold for now:
%
%\begin{center}
%*Authors contributed equally to this work
%\end{center}

\vspace{0.2in} % Some whitespace between sections



%\section{\centerline{CONFERENCE PROCEEDINGS}} 
%\vspace{-0.1in} % Some whitespace between sections
%\begin{center}
%{\bf 1 accepted conference proceeding}
%\end{center}
%
%\vspace{10pt} % Gap between title and text
%\begin{etaremune}
%  \item \me; \cy;
%        ``\emph{Parallel Shift-Invert Spectrum Slicing on Distributed Architectures with GPU Accelerators}"
%        in \emph{Proceedings of the 49th International Conference on Parallel Processing (ICPP'20)}, \textbf{2020},
%        Forthcoming.
%\end{etaremune}
%
%\vspace{0.2in} % Some whitespace between sections

%\section{\centerline{CURRENT SOFTWARE CITATIONS}} 
%\vspace{15pt} % Gap between title and text
%
%\begin{etaremune}
%  \item \me; \emph{HAXX: Hamilton's Quaternion Algebra for CXX}, \url{http://github.com/wavefunction91/HAXX},
%  \textbf{2019}.
%
%  \item \xsli; Valeev,~E.~F.; \me; Ding,~F.; Liu,~H.; Goings,~J.~J.; Petrone,~A.; Lestrange,~P.;
%        \emph{Chronus Quantum, Beta Version}, \url{http://www.chronusquantum.org}, \textbf{2019}.
%
%\item Frisch,~M.~J.; Trucks,~G.~W.; Schlegel,~H.~B.; Scuseria,~G.~E.; Robb,~M.~A.;
%      Cheeseman,~J.~R.; Scalmani,~G.; Barone,~V.; Petersson,~G.~A.; Nakatsuji,~H.; \xsli;
%      Caricato,~M.; Marenich,~A.; Bloino,~J.; Janesko,~B.~G.; Gomperts,~R.; Mennucci,~B.;
%      Hratchian,~H.~P.; Izmaylov,~A.~F.; Sonnenberg,~J.~L.; \me; Ding,~F.; Lipparini,~F.;
%      Egidi,~F.; Goings,~J.; Peng,~B.; Petrone,~A.; Ortiz,~J.~V.; Zakrzewski,~V.~G.; Gao,~J.;
%      Rega,~N.; Zheng,~G.; Liang,~W.; Hada,~M.; Ehara,~M.; Toyota,~K.; Fukuda,~R.; Hasegawa,~J.;
%      Ishida,~M.; Nakajima,~T.; Honda,~Y.; Kitao,~O.; Nakai,~H.; Vreven,~T.; Throssell,~K.; Montgomery~Jr.,~J.~A.;
%      Peralta,~J.~E.; Ogliaro,~F.; Bearpark,~M.; Heyd,~J.~J.; Brothers,~E.; Kudin,~K.~N.; Staroverov,~V.~N.; Keith,~T.;
%      Kobayashi,~R.; Normand,~J.; Raghavachari,~K.; Rendell,~A.; Burant,~J.~C.; Iyengar,~S.~S.;
%      Tomasi,~J.; Cossi,~M.; Millam,~J.~M.; Klene,~M.; Adamo,~C.; Cammi,~R.;
%      Ochterski,~J.~W.; Martin,~R.~L.; Morokuma,~K.; Farkas,~O.; Foresman,~J.~B.; and Fox~,~D.~J.;
%      \emph{Gaussian 16, A.03},
%      Gaussian, Inc., Wallingford CT, \textbf{2016}.
%\end{etaremune}
%
%\vspace{0.2in} % Some whitespace between sections
%----------------------------------------------------------------------------------------
%	HONORS SECTION
%----------------------------------------------------------------------------------------

\section{\centerline{HONORS}} 

\vspace{-5pt} % Reduce space between section title and contents

\begin{center}
CCG Excellence Award for Graduate Students \hfill The Chemical Computing Group (2017) \\
MolSSI Software Fellow \hfill Molecular Sciences Software Institute (2017-2018) \\
Lloyd E. and Florence M. West Fellowship in Chemistry \hfill Lloyd E. and Florence M. West (2016) \\
%Early Bird Research Assistantship (EBRA) \hfill University of Washington (2013) \\
Excellence in Chemistry Graduate Fellowship Award (ECGFA) \hfill University of Washington (2013)\\
%Provost Scholar \hfill Indiana University of Pennsylvania (2013)
\end{center}

%----------------------------------------------------------------------------------------

\vspace{0.2in} % Some whitespace between sections
%----------------------------------------------------------------------------------------
%	SERVICE SECTION
%----------------------------------------------------------------------------------------
\section{\centerline{PROFESSIONAL SERVICE}} 

%\vspace{-5pt} % Reduce space between section title and contents

%\begin{center}
%Reviewer, \emph{Molecular Physics}                \hfill (2023) \\
%Review Panelist, \emph{NSF Office of Advanced Cyberinfrastructure}, \hfill (2022) \\
%Reviewer, \emph{The Journal of Chemical Physics}                \hfill (2020) \\
%Reviewer, \emph{The International Journal of Quantum Chemistry} \hfill (2020) \\
%Reviewer, \emph{Computer Physics Communications}                \hfill (2020) \\
%Minisymposium Organizer, \emph{SIAM Conference on Parallel Processing for Scientific Computing} \hfill (2020) \\ 
%Minisymposium Organizer, \emph{SIAM Conference on Computational Science and Engineering} \hfill (2019,2021,2023) \\ 
%Reviewer, \emph{The Journal of Chemical Theory and Computation} \hfill (2019) \\
%Reviewer, \emph{Journal of Computational Physics}               \hfill (2019) 
%\end{center}
\vspace{20pt}
\centerline{\bf Leadership Activities}
\vspace{-20pt}
\begin{center}
\emph{Journal of Chemical Theory and Computation Early Career Board} \hfill (2024-Present)\\
\emph{NWChemEx Executive Board} \hfill (2023-Present)
\end{center}


\centerline{\bf Journal Review}
\vspace{-20pt}
\begin{center}
  \emph{Molecular Physics}, 
  \emph{The Journal of Chemical Physics}, 
  \emph{The International Journal of Quantum Chemistry}, 
  \emph{Computer Physics Communications},
  \emph{The Journal of Chemical Theory and Computation},
  \emph{Journal of Computational Physics},
  \emph{The Journal of Physical Chemistry}, 
  \emph{The Journal of Physical Chemistry A} 
\end{center}

\centerline{\bf Proposal Review}
\vspace{-15pt}
\begin{center}
\emph{NSF Office of Advanced Cyberinfrastructure} \hfill (2022, 2023) \\
\emph{DOE Office of Science, Office of Basic Energy Sciences} \hfill (2023) \\
\emph{Natural Sciences and Engineering Research Council of Canada} \hfill (2023)
\end{center}

\centerline{\bf Conference Symposium Organization}
\vspace{-15pt}
\begin{center}
\emph{SIAM Conference on Computational Science and Engineering}        \hfill (2019, 2021, 2023) \\
\emph{SIAM Conference on Parallel Processing for Scientific Computing} \hfill (2020, 2022, 2024)
\end{center}

%----------------------------------------------------------------------------------------
\vspace{0.2in} % Some whitespace between sections

\section{\centerline{PRESENTATIONS}} 
\begin{center}
* Indicates an invited presentation
\end{center}

\vspace{15pt} % Gap between title and text
\begin{etaremune}
  \item * \presentation%
        {\me}%
        {Massively Parallel Selected Configuration Interaction for First-Principles Quantum Chemistry Applications}%
        {Society for Industrial and Applied Mathematics Conference on Parallel Processing for Scientific Computing (SIAM-PP24)}%
        {Baltimore, MD}%
        {2024}%
        {Oral Presentation} 
  \item \presentation%
        {\me}%
        {Revisiting Pseudospectral Methods for Hybrid Kohn-Sham Density Functional Theory in the Age of Exascale Computing}%
        {American Phyiscal Society March Meeting 2023}%
        {Las Vegas, NV}%
        {2023}%
        {Oral Presentation} 
  \item * \presentation%
        {\me}%
        {Perspective on High-Performance Algorithms for Eigenvalue Problems in Physical Simulations}%
        {Society for Industrial and Applied Mathematics Conference on Computational Science and Engineering (SIAM-CSE23)}%
        {Amsterdam, The Netherlands}%
        {2023}%
        {Oral Presentation} 
  \item * \presentation%
        {\me}%
        {Modular Design Patterns for the Sustainable Development of High-Performance Computational Chemistry Software}%
        {Basic Energy Sciences Workshop on Sustainable Computational Chemistry Software Development, Integration, and Application}%
        {Seattle, WA}%
        {2022}%
        {Oral Presentation} 
  \item * \presentation%
        {\me}%
        {Low-Order, Scalable Eigensolvers for Electronic Structure Simulations Based on Spectrum Slicing}%
        {CECAM Workshop on Challenges and Advances in Solving Eigenproblems for Electronic Structure Theory}%
        {Lausanne, Switzerland}%
        {2022}%
        {Oral Presentation} 
  \item \presentation%
        {\me}%
        {Revisiting Pseudospectral Methods for Hybrid Kohn-Sham Density Functional Theory in the Age of Exascale Computing}%
        {263rd American Chemical Society National Meeting \& Exposition}%
        {San Diego, CA}%
        {2022}%
        {Oral Presentation} 
  \item * \presentation%
        {\me}%
        {Revisiting Pseudospectral Methods for Hybrid Kohn-Sham Density Functional Theory in the Age of Exascale Computing}%
        {Society for Industrial and Applied Mathematics Conference on Parallel Processing for Scientific Computing (SIAM-PP22)}%
        {Virtual}%
        {2022}%
        {Oral Presentation} 
  \item \presentation%
        {\me}%
        {Developing (Semi)numerical Methods for Hybrid Kohn-Sham Density Functional Theory in the Age of Exascale Computing}%
        {The International Chemical Congress of Pacific Basin Societies (Pacifichem '21)}%
        {Virtual}%
        {2021}%
        {Oral Presentation} 
  \item * \presentation%
        {\me}%
        {Parallel Shift-Invert Spectrum Slicing for Symmetric Self-Consistent Eigenvalue Computation}%
        {Society for Industrial and Applied Mathematics Conference on Computational Science and Engineering (SIAM-CSE21)}%
        {Virtual}%
        {2021}%
        {Oral Presentation} 
  \item \presentation%
        {\me}%
        {Parallel Shift-Invert Spectrum Slicing on Computing Clusters with GPU Accelerators}%
        {49th International Conference on Parallel Processing (ICPP20)}%
        {Virtual}%
        {2020}%
        {Oral Presentation} 
  \item \presentation%
        {\me; \cy}%
        {Parallel Shift-Invert Spectrum Slicing for Symmetric Self-Consistent Eigenvalue Computation}%
        {Society for Industrial and Applied Mathematics Conference on Parallel Processing for Scientific Computing (SIAM-PP20)}%
        {Seattle, WA}%
        {2020}%
        {Oral Presentation} 
  \item \presentation%
        {\me; Van Beeuman, R.; \cy; \xsli}%
        {A General Model Order Reduction Scheme for the Evaluation of Spectroscopic Properties and Excited States}%
        {American Physical Society March Meeting 2019}%
        {Boston, MA}%
        {2019}%
        {Oral Presentation} 
  \item * \presentation%
        {\me}%
        {On the High Performance Implementation of Quaternionic Matrix Operations}%
        {Society for Industrial and Applied Mathematics Conference on Computational Science and Engineering (SIAM-CSE19)}%
        {Spokane, WA}%
        {2019}%
        {Oral Presentation} 
  \item \presentation%
        {\me; Van Beeuman, R.; \cy; \xsli}%
        {A Novel Model Reduction Algorithm for the Efficient Evaluation of Molecular Response Properties}%
        {254th American Chemical Society National Meeting \& Exposition}%
        {Washington, DC}%
        {2017}%
        {Poster} 
  \item * \presentation%
        {\me}%
        {Studying Semi--Classical Molecular Light--Matter Interaction through Time--Dependent Density Function Theory}%
        {High Performance Computing Seminar}%
        {University of Washington, Seattle, WA}%
        {2017}%
        {Oral Presentation} 
  \item \presentation%
        {\me; Goings, J.J.; \xsli}%
        {Accelerating Real--Time Time--Dependent Density Functional Theory with a Chebyshev Expansion of the Quantum Propagator}%
        {Theory and Applications of Computational Chemistry (TACC) 2016}%
        {Seattle, WA}%
        {2016}%
        {Poster} 
%  \item Egidi F.; \me; \xsli;
%        ``\emph{Electronic Structure Methods for Relativistic Effects in Excited States}";
%	Low Scaling and Unconventional Electronic Structure Theory (LUEST) 2016, Telluride, CO.
%	\textbf{2016}. Poster.
  \item \presentation%
        {\me; Yang, W.; \xsli}%
        {Moving past the particle-hole description of excited states: Affordable methodologies}%
        {The International Chemical Congress of Pacific Basin Societies (Pacifichem '15)}%
        {Honolulu, HI}%
        {2015}%
        {Oral Presentation} 
%  \item Scalmani, G.; Frisch, M.; \me; \xsli;
%	``\emph{On the analytical second derivatives of the excited state energy in the 
%	        framework of Time-Dependant Density Functional Theory}";
%	248h American Chemical Society National Meeting \& Exposition, San Francisco, CA.
%	\textbf{2014}. COMP: Quantum Chemistry Symposium. Oral Presentation.
  \item \presentation%
        {\me; \ko; Ondrechen, M.J.}%
        {Computational approach to the prediction of enzyme specificities}%
        {245th American Chemical Society National Meeting \& Exposition}%
        {New Orleans, LA}%
        {2013}%
        {Poster} 
  \item \presentation%
        {\me; \ko}%
        {Prediction of Relative Activities of Enzymes using Computed Chemical Properties}%
        {American Chemical Society Student Member Symposium}%
        {Duquesne University, Duquesne, PA}%
        {2012}%
        {Poster} 
  \item \presentation%
        {\me; \ko}%
        {Prediction of Relative Activities of Enzymes using Computed Chemical Properties}%
        {Undergraduate Research Forum}%
        {Indiana University of Pennsylvania. Indiana, PA}%
        {2012}%
        {Poster} 
%  \item Ford, J.; \ko; Mintmier, B.; Machovia, T.; Kang, M.; Owens, A.; \me;
%	``\emph{Undergraduate Research in Biomass Utilization Symposium}",
%	Pennsylvania Association of the Council of Trustees Conference, Indiana University of Pennsylvania.
%	Indiana, PA. \textbf{2011}. Oral Presentation.
\end{etaremune}
%----------------------------------------------------------------------------------------

\vspace{0.2in} % Some whitespace between sections

%----------------------------------------------------------------------------------------
%	MEMBERSHIPS SECTION
%----------------------------------------------------------------------------------------

\section{\centerline{MEMBERSHIPS}} 

\vspace{-5pt} % Reduce space between section title and contents

\begin{center}
Alpha Chi Sigma ($\mathrm{AX}\Sigma$), $\Gamma\mathrm{T}$ Chapter, Professional Chemistry Fraternity\\
American Chemical Society (ACS), Computers in Chemistry Division\\
American Physical Society (APS) \\
Society for Industrial and Applied Mathematics (SIAM), Activity Group on Computational Science and Engineering 
\end{center}

%----------------------------------------------------------------------------------------

%\vspace{0.2in} % Some whitespace between sections
%
%%----------------------------------------------------------------------------------------
%%	INTERESTS SECTION
%%----------------------------------------------------------------------------------------
%
%\section{\centerline{INTERESTS}} 
%
%\vspace{-5pt} % Reduce space between section title and contents
%
%\begin{center}
%World percussion, kayaking, hiking, climbing, Linux kernel development
%\end{center} 

%----------------------------------------------------------------------------------------

\end{resume} 
\end{document}
