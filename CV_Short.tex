%%%%%%%%%%%%%%%%%%%%%%%%%%%%%%%%%%%%%%%%%
% Long Professional Curriculum Vitae
% LaTeX Template
% Version 1.1 (9/12/12)
%
% This template has been downloaded from:
% http://www.latextemplates.com
% Original author:
% Rensselaer Polytechnic Institute (http://www.rpi.edu/dept/arc/training/latex/resumes/)
%
% Important note:
% This template requires the res.cls file to be in the same directory as the
% .tex file. The res.cls file provides the resume style used for structuring the
% document.
%
%%%%%%%%%%%%%%%%%%%%%%%%%%%%%%%%%%%%%%%%%

%----------------------------------------------------------------------------------------
%	PACKAGES AND OTHER DOCUMENT CONFIGURATIONS
%----------------------------------------------------------------------------------------

\let\latexnofiles\nofiles
\let\nofiles\relax
\documentclass[10pt]{res} % Use the res.cls style, the font size can be changed to 11pt or 12pt here

\usepackage{helvet} % Default font is the helvetica postscript font
%\usepackage{newcent} % To change the default font to the new century schoolbook postscript font uncomment this line and comment the one above
\usepackage{etaremune}
\usepackage{hyperref}
\usepackage{enumitem}

\newsectionwidth{0pt} % Stops section indenting

% Name shortcuts
\newcommand*\me[0]{{\bf Williams--Young,~D.~B.}}
\newcommand*\xsli[0]{Li,~X.}
\newcommand*\cy[0]{Yang,~C.}
\newcommand*\bdj[0]{de~Jong,~W.A.}
\newcommand*\ko[0]{Ko,~J.}

%\newcommand{\invited}{\item[$*$\theenumi.]}
\newcommand{\invited}{\refstepcounter{enumi}\item[$*$\theenumi.]}
\newcommand{\presentation}[6]{%
{#1}; ``\emph{#2}"; {#3}; {#4}; \textbf{#5}; {#6}.%
}

\begin{document}

%----------------------------------------------------------------------------------------
%	YOUR NAME AND ADDRESS(ES) SECTION
%----------------------------------------------------------------------------------------

\name{Dr. David Williams--Young, Ph.D\\ \\} % Your name at the top

% If you don't want one of the addresses, simply remove all the text in the first or second \address{} bracket

\address{Research Scientist (Career),\\
Applied Commputing for Scientific Discovery Group,\\
Applied Mathematics and Computational Research Division,\\ 
Lawrence Berkeley National Laboratory\\ 
+1 (510) 495-2189\\
dbwy@lbl.gov} % Your address 1

\address{ {\bf Address} \\ 
Lawrence Berkeley National Laboratory\\ 
50F-1645 (office),  
MS 50F-1650 (mail)\\
1 Cyclotron Road, 
Berkeley, CA 94720\\
%1241 Homestead Ave \\ 
%Apt 222 \\
%Walnut Creek, CA, USA 94598
} % Your address 2

%----------------------------------------------------------------------------------------

\begin{resume}

%%----------------------------------------------------------------------------------------
%%	OBJECTIVE SECTION
%%----------------------------------------------------------------------------------------
%
%\section{\centerline{OBJECTIVE}}
%
%\vspace{8pt} % Gap between title and text
%
%To obtain a research position relating to high--performance scientific computing and numerical
%linear algebra and their interplay with methods development in electronic structure theory.


%----------------------------------------------------------------------------------------
%       RESEARCH INTERESTS
%----------------------------------------------------------------------------------------

%\section{\centerline{RESEARCH INTERESTS}}
%\begin{itemize} \itemsep -2pt
%%  \item Development of low--scaling relativistic electronic structure methods to describe
%%  strongly correlated systems, such as transition metal complexes commonly found in
%%  the vicinity of the active sites of enzymes.
%%  \item Application of abstract mathematical paradigms, such as algebraic topology, to
%%  develop elegant and novel solutions to the problems that arise in ab initio electronic
%%  structure theory.
%%  \item Development of high--performance algorithms to treat the electronically non--adiabatic
%%  dynamics of quantum molecular systems.
%  \item Development of high--performance and reduced scaling electronic structure
%  methods on emerging architectures (GPUs, FPGAs, etc).
%  \item Development of novel field-theoretic methods for the elucidation of relativistic
%  electronic structure in superheavy elements.
%  \item Development of novel quantum algorithms for the simulation of the relativistic
%  many-body problem.
%  \item The intersection of high-performance computing and quantum information science.
%  \item Algorithmic development for high--performance, sparse linear algebra software
%  (Krylov solvers).
%  %\item Application of algebraic topology and differential geometry to develop elegant and 
%  %novel solutions to the problems that arise in ab initio electronic structure theory.
%\end{itemize}

%\section{\centerline{RESEARCH SUMMARY}}
%\begin{itemize} \itemsep -2pt
%  \item \textbf{Relativistic Excited State Electronic Structure Theory}, \emph{2014 -- Present}. 
%  A primary focus of my current and past research has been the development of efficient relativistic
%  excited state methods for electronic structure theory. Recently, I have extended the
%  particle--particle Tamm--Dancoff approximation, a method well known in \emph{ab initio}
%  nuclear structure theory, to two--component relativistic Hamiltonians. Using this method,
%  I was able to describe, with excellent accuracy, the fine--structure splitting of the excited
%  states of a set of atomic and molecular systems [\emph{J. Chem. Theor. Comp.}, \textbf{2016}, 12(11), 5379--5384].
%  I am also currently developing relativistic extensions for various other post-SCF methods such as
%  Coupled Cluster and Configuration Interaction.\\
%
%  \item \textbf{Molecular Dynamics}, \emph{2015 -- Present}. Recently, much of my research effort
%  has been directed to the development of novel and efficient electronic structure methods to describe
%  the electronic and nuclear dynamics of molecular systems in an \emph{ab initio} manner. I have developed
%  a revised algorithm for the evaluation of the quantum propagator of time--dependent density functional
%  theory (TDDFT) through the Chebyshev expansion of the matrix exponential [\emph{J. Chem. Theor. Comp.}, \textbf{2016}, 12(11) 5333--5338].
%  In principle, this development makes it possible to achieve linear--scaling electronic dynamics simulations
%  through the use of sparse matrix manipulation.\\
%  \\
%  Further, in collaboration with my colleagues in the Li research group, I have been involved with several 
%  exciting developments in the field of nonadiabatic excited state nuclear dynamics. The key to the usability
%  of any nonadiabatic molecular dynamics method is the underlying efficiency of the algorithms used to obtain
%  the required physical moieties, such as gradients and nonadiabatic couplings, throughout the simulation.
%  Much of my work in this field has been in the development of extremely efficient and scalable methods to
%  obtain these quantities. My work has facilitated the development of several methodological developments in the
%  field, including direct \emph{ab initio} rare event sampling through the meta--surface hopping method
%  [\emph{J. Chem. Theor. Comp.}, \textbf{2016}, 12(3), 935--945], and the analysis of transient vibrational signatures
%  to describe confomational changes through conformational changes 
%  [\emph{J. Chem. Phys. Comm.}, \textbf{2016}, 7, 4501--4508].
%\end{itemize}
%
%\section{\centerline{FUTURE RESEARCH STATEMENT}}
%  My plan for future research directions involves the combination of my previous research topics through
%  the application of relativistic hamiltonians to describe the nonadibatic behavior of molecular dynamics. To properly
%  treat nonadiabatic transitions between electronic states of differing spin multiplicities, such as intersystem crossings
%  and phosphorescence, a proper treatment of spin--orbit coupling is of utmost importance. While several approaches have
%  been to describe spin--orbit perturbatively in the non--relativistic regime, the only method to properly describe 
%  the effects of spin--orbit lie in relativistic treatments. Through this research direction, I will be able to better model
%  many chemical phenomena, such as charge transfer and reaction dynamics, which are often mediated by spin--forbidden processes.

\vspace{0.2in} % Some whitespace between sections

%----------------------------------------------------------------------------------------
%	PROFESSIONAL EXPERIENCE SECTION
%----------------------------------------------------------------------------------------

\section{\centerline{PROFESSIONAL EXPERIENCE}} 

{\sl \bf Research Scientist (Career)} \dotfill February 2021 -- Present \\
{\sl \bf Postdoctoral Fellow} \dotfill July 2018 -- February 2021\\
Applied Mathematics and Computational Research Division \\
Lawrence Berkeley National Laboratory, 
Berkeley, CA

{\sl \bf Graduate Research Assistant} \dotfill July 2013 -- July 2018 \\
{\sl \bf Graduate Teaching Assistant} \dotfill September 2013 -- July 2018 \\
Department of Chemistry \\
University of Washington, Seattle, WA 
%\begin{itemize} \itemsep -2pt
%  \item Provided supplementary lecture instruction for the undergraduate General Chemistry course series
%    as well as full instruction for the associated lab coursework. This job entailed full responsibility
%    for two sections of thirty students during a supplemental lecture series referred to as a quiz section.
%    I was responsible for the evaluation of students' performance both in lecture coursework (exams, homework assignments)
%    as well as the lab (lab reports).
%  \item Provided supplementary lecture instruction for the final lecture course (in a series of three) of
%    the undergraduate Organic Chemistry course series. Primarily, I was responsible for evaluating students' performance on
%    Exams. In addition, I was responsible for making myself available for course related questions through
%    structured office hours.
%  \item Aided in the development of new course material for the second lecture course (in a series of two),
%    of the graduate Quantum Chemistry course series. The developed course material focused on the practical
%    development of basic quantum chemical methods, such as Hartree-Fock and Density Functional Theory, using
%    the MATLAB development environment. The students developed, from scratch, a working implementation
%    of these methods with the aid of provided course material developed by my self and the primary course
%    instructor. For this course, I also had the primary responsibility for evaluating students' progress
%    through the grading of exams and assigned course work.
%\end{itemize}

%{\sl \bf Undergraduate Research Assistant} \dotfill September 2011 -- May 2013 \\
%Department of Chemistry \\
%Indiana University of Pennsylvania, Indiana, PA

%{\sl Chemistry Tutor} \hfill August 2010 -- May 2011 \\
%Indiana University of Pennsylvania Disability Services, Indiana, PA 
%\begin{itemize}
%  \item Provided supplemental course instruction for the General and Organic Chemistry
%    series for students with disabilities. This primarily entailed meeting with students
%    individually on a weekly basis to aid in their understanding of course material and 
%    to ensure that they were not falling behind in coursework.
%\end{itemize}

%{\sl Information Technology Technician} \hfill August 2009 -- January 2010 \\
%Central Michigan University Information Technology, Mt. Pleasant, MI
%%\begin{itemize}
%%  \item Primarily responsible for solving network connectivity issues for incoming students
%%    at Central Michigan University.
%%\end{itemize}


%----------------------------------------------------------------------------------------
%	EDUCATION SECTION
%----------------------------------------------------------------------------------------
\section{\centerline{EDUCATION}} 

{\sl\bf  Doctor of Philosophy} (Ph.D.), Chemistry \dotfill May 2018 \\ 
University of Washington, Seattle, WA   
%Adviser: Dr. Xiaosong Li \\
%Dissertation: \emph{Towards Efficient and Scalable Electronic Structure Methods for 
%the Treatment of Relativistic Effects and Molecular Response}
 
{\sl\bf Bachelor of Science} (B.S., \emph{Magna Cum Laude}), Chemistry, Mathematics \dotfill May 2013  \\ 
Indiana University of Pennsylvania, Indiana, PA \\
%Adviser: Dr. Jaeju Ko


\vspace{-0.2in}

%----------------------------------------------------------------------------------------
%	COMPUTER SKILLS SECTION
%----------------------------------------------------------------------------------------

\section{\centerline{SOFTWARE}}
\vspace{0.1in}

I am the primary developer of the following open-source software packages:
\begin{itemize} \itemsep -2pt
  \item {\bf GauXC}:   Enabling high-performance density functional theory calculations on exascale architectures
  \item {\bf MACIS}:   Massively parallel selected configuration interaction methods
  \item {\bf ExchCXX}: GPU-accelerated library for the evaluation of exchange-correlation functionals
\end{itemize}

I have contributed to the development of the following software packages:
The Chronus Quantum (ChronusQ) Software Package,
NWChemEx,
Massively Parallel Quantum Chemistry (MPQC4),
TiledArray,
Gaussian

%----------------------------------------------------------------------------------------

%\vspace{0.2in} % Some whitespace between sections

%----------------------------------------------------------------------------------------
%	PUBLICATIONS SECTION
%----------------------------------------------------------------------------------------

\section{\centerline{SELECTED PUBLICATIONS}} 

\begin{itemize}[leftmargin=*]
  \item \me; Tubman, N.M.; Mejuto-Zaera, C.; \bdj;
        ``\emph{A Parallel, Distributed Memory Implementation of the Adaptive 
                Sampling Configuration Interaction Method}";
        \emph{J. Chem. Phys.}; \textbf{2023}. 158, 214109. 
  \item \me; Yuwono, S.; DePrince III, A.E.; \cy; 
        ``\emph{Approximate Exponential Integrators for Time-Dependent Equation-of-Motion Coupled Cluster Theory}”
	\emph{J. Chem. Theory Comput.}; \textbf{2023}. 2023, 19, 24, 9177–9186.
  \item \me; Asadchev, A.; Popovici, D.T; Clark, D.; Waldrop, J.; Windus, T.L.;
        Valeev, E.F.; \bdj;
        ``\emph{Distributed Memory, GPU Accelerated Fock Construction for Hybrid, Gaussian 
                Basis Density Functional Theory}";
        \emph{J. Chem. Phys.}; \textbf{2023}. 158, 234104.
  \item \me; Bagusetty, A.; \bdj; Doerfler, D.; vam Dam, H.J.J.; Vazquez--Mayagoitia, A.;
        Windus, T.L.; Yang, C.;
        ``\emph{Achieving Performance Portability in Gaussian Basis Set Density Functional Theory 
                on Accelerator Based Architectures}"; \emph{Parallel Computing},
        \textbf{2021}, 108, 102829.
%  \item Di Felice, R.; Mayes, M.; Richard, R.; \me; Chan, G.K.L; \bdj; 
%        Govind, N.; Head-Gordon, M.; Hermes, M.; Kowalski, K.; Li, X.; Lischka, H.; Mueller, K.; 
%	Mutlu, E.; Niklasson, A.; Pederson, M.; Peng, B.; Shepard, R.; Valeev, E.; van Schilfgaarde, M.; 
%	Vlaisavljevich, B.; Windus, T.; Xantheas, S.; Zhang, X.; Zimmerman, P.;
%	``\emph{A Perspective on Sustainable Computational Chemistry Software Development and Integration}"
%	\emph{J. Chem. Theory Comput.}; \textbf{2023}. 19, 20, 7056–7076.
%  \item Kowalski, K.; Bair, R.; Bauman, N.P.; Boschen, J.S.; Bylaska, E.J.; Daily, J.; \bdj; 
%        Dunning, T.; Govind, N.; Harrison, R.J.; Keceli, M.; Keipert, K.; Krishnamoorthy, S.; Kumar, S.;
%        Mutlu, E.; Palmer, B.; Panyala, A.; Peng, B.; Richard, R.M.; Straatsma, T.P.; Sushko, P.; Valeev, E.F.;
%        Valiev, M.; van Dam, H.J.J.; Waldrop, J.M.; \me; \cy; Zalewski, M.; Windus, T.L.;
%       ``\emph{From NWChem to NWChemEx: Evolving with the Computational Chemistry Landscape}"; \emph{Chemical Reviews},
%       \textbf{2021}, 121(8), 4962--4998.
%  \item \me; Petrone, A.; Sun, S.; Stetina, T.F.; Lestrange, P.; Hoyer, C.E.; 
%        Nascimento, D.R.; Koulias, L.; Wildman, A.; Kasper, J.; Goings, J.J.; 
%        Ding, F.; DePrince, A.E.; Valeev, E.F.; \xsli;
%        ``\emph{The Chronus Quantum (ChronusQ) Software Package}"
%        \emph{WIREs Comput. Mol. Sci.} \textbf{2019}, e1436.
\end{itemize}


\section{\centerline{CURRENT PROJECTS}} 

\vspace{5pt}

{\sl\bf Scalable Predictive Methods for Excitations and Correlated Phenomena} \dotfill 2023-2025\\
DOE-BES (Computational Chemical Sciences - CCS, LAB 17-1775) 

This project focuses on the development of software and algorithms for the
simulation of molecular excited states with unprecedented predictive power and
orders-of-magnitude greater computational performance than current methods.
This project funds the development of the Many-body Adaptive Configuration
Interaction Suite (MACIS) which will serve as the primary development platform
for the work considered in this proposal.

\vspace{5pt}

{\sl\bf Relativistic Quantum Dynamics in the Non-Equilibrium Regime} \dotfill 2021-2026\\
DOE-BES/ASCR (Scientific Discovery through Advacned Computing - SciDAC, DE-SC0022263)

This project focuses on the development of high-performance methods to the
simulation of relativistic temporal dynamics in complicated many-body quantum
systems. Through this project we have developed approximate (Krylov subspace),
massively parallel techniques for the propagation of correlated quantum systems
with non-hermitian Hamiltonians, such as those arising from coupled cluster
theory. This project is not related to the work considered in this proposal.


\section{\centerline{CURRENT AND RECENT COLLABORATORS}} 

\vspace{5pt}

Karol Kowalski \dotfill (Pacific Northwest National Lab)\\
Niri Govind \dotfill (Pacific Northwest National Lab)\\
Bo Peng \dotfill (Pacific Northwest National Lab)\\
Soritis Xantheas \dotfill (Pacific Northwest National Lab)\\
Theresa Windus \dotfill (Iowa State University / Ames National Lab)\\
Ryan Richards \dotfill (Iowa State University / Ames National Lab)\\
Hubertus Van Dam \dotfill (Brookhaven National Lab)\\
Murat Keceli \dotfill (Argonne National Lab)\\
Alvaro Vazquez Mayagoitia \dotfill (Argonne National Lab)\\
Albert Eugene DePrince III \dotfill (Florida State University)\\
Edward Valeev \dotfill (Virginia Tech)\\
Xiaosong Li \dotfill (University of Washington)\\


%\section{\centerline{HONORS}} 
%
%\vspace{-5pt} % Reduce space between section title and contents
%
%\begin{center}
%Board Member \hfill Journal of Chemical Theory and Computation Early Career Board (2023) \\
%CCG Excellence Award for Graduate Students \hfill The Chemical Computing Group (2017) \\
%MolSSI Software Fellow \hfill Molecular Sciences Software Institute (2017-2018) \\
%Lloyd E. and Florence M. West Fellowship in Chemistry \hfill Lloyd E. and Florence M. West (2016) \\
%Excellence in Chemistry Graduate Fellowship Award (ECGFA) \hfill University of Washington (2013)
%\end{center}
%
%%----------------------------------------------------------------------------------------
%%	SERVICE SECTION
%%----------------------------------------------------------------------------------------
%\section{\centerline{PROFESSIONAL SERVICE}} 
%
%%\vspace{-5pt} % Reduce space between section title and contents
%
%%\begin{center}
%%Reviewer, \emph{Molecular Physics}                \hfill (2023) \\
%%Review Panelist, \emph{NSF Office of Advanced Cyberinfrastructure}, \hfill (2022) \\
%%Reviewer, \emph{The Journal of Chemical Physics}                \hfill (2020) \\
%%Reviewer, \emph{The International Journal of Quantum Chemistry} \hfill (2020) \\
%%Reviewer, \emph{Computer Physics Communications}                \hfill (2020) \\
%%Minisymposium Organizer, \emph{SIAM Conference on Parallel Processing for Scientific Computing} \hfill (2020) \\ 
%%Minisymposium Organizer, \emph{SIAM Conference on Computational Science and Engineering} \hfill (2019,2021,2023) \\ 
%%Reviewer, \emph{The Journal of Chemical Theory and Computation} \hfill (2019) \\
%%Reviewer, \emph{Journal of Computational Physics}               \hfill (2019) 
%%\end{center}
%~\\
%\centerline{\bf Journal Review}
%\vspace{-20pt}
%\begin{center}
%  \emph{Molecular Physics}, 
%  \emph{The Journal of Chemical Physics}, 
%  \emph{The International Journal of Quantum Chemistry}, 
%  \emph{Computer Physics Communications},
%  \emph{The Journal of Chemical Theory and Computation},
%  \emph{Journal of Computational Physics},
%  \emph{The Journal of Physical Chemistry}, 
%  \emph{The Journal of Physical Chemistry A} 
%\end{center}
%
%\centerline{\bf Proposal Review}
%\vspace{-15pt}
%\begin{center}
%\emph{NSF Office of Advanced Cyberinfrastructure} \hfill (2022, 2023) \\
%\emph{DOE Office of Science, Office of Basic Energy Sciences} \hfill (2023) \\
%\emph{Natural Sciences and Engineering Research Council of Canada} \hfill (2023)
%\end{center}
%
%\centerline{\bf Conference Symposium Organization}
%\vspace{-15pt}
%\begin{center}
%\emph{SIAM Conference on Computational Science and Engineering}        \hfill (2019, 2021, 2023) \\
%\emph{SIAM Conference on Parallel Processing for Scientific Computing} \hfill (2020)
%\end{center}


\end{resume} 
\end{document}
